\documentclass{article}

\usepackage{fancyhdr}
\usepackage{extramarks}
\usepackage{amsmath}
\usepackage{amsthm}
\usepackage{amsfonts}
\usepackage{amssymb}
\usepackage{graphicx}
\usepackage{caption,subcaption}

%
% Basic Document Settings
%

\topmargin=-0.45in
\evensidemargin=0in
\oddsidemargin=0in
\textwidth=6.5in
\textheight=9.0in
\headsep=0.25in

\linespread{1.1}

\pagestyle{fancy}
\lhead{\hmwkAuthorName}
\chead{\hmwkClass:\ \hmwkTitle}
\rhead{\firstxmark}
\lfoot{\lastxmark}
\cfoot{\thepage}

\renewcommand\headrulewidth{0.4pt}
\renewcommand\footrulewidth{0.4pt}

\setlength\parindent{0pt}

%
% Create Problem Sections
%

\newcommand{\enterProblemHeader}[1]{
    \nobreak\extramarks{}{Problem \arabic{#1} continued on next page\ldots}\nobreak{}
    \nobreak\extramarks{Problem \arabic{#1} (continued)}{Problem \arabic{#1} continued on next page\ldots}\nobreak{}
}

\newcommand{\exitProblemHeader}[1]{
    \nobreak\extramarks{Problem \arabic{#1} (continued)}{Problem \arabic{#1} continued on next page\ldots}\nobreak{}
    \stepcounter{#1}
    \nobreak\extramarks{Problem \arabic{#1}}{}\nobreak{}
}

\setcounter{secnumdepth}{0}
\newcounter{partCounter}
\newcounter{homeworkProblemCounter}
\setcounter{homeworkProblemCounter}{1}
\nobreak\extramarks{Problem \arabic{homeworkProblemCounter}}{}\nobreak{}

%
% Homework Problem Environment
%
% This environment takes an optional argument. When given, it will adjust the
% problem counter. This is useful for when the problems given for your
% assignment aren't sequential. See the last 3 problems of this template for an
% example.
%
\newenvironment{homeworkProblem}[1][-1]{
    \ifnum#1>0
        \setcounter{homeworkProblemCounter}{#1}
    \fi
    \section{Problem \arabic{homeworkProblemCounter}}
    \setcounter{partCounter}{1}
    \enterProblemHeader{homeworkProblemCounter}
}{
    \exitProblemHeader{homeworkProblemCounter}
}

%
% Homework Details
%   - Title
%   - Due date
%   - Class
%   - Section/Time
%   - Instructor
%   - Author
%

\newcommand{\hmwkTitle}{Homework 2}
\newcommand{\hmwkDueDate}{April 16, 2015}
\newcommand{\hmwkClass}{Amath 586}
\newcommand{\hmwkAuthorName}{Brian de Silva}

%
% Title Page
%

\title{
    \vspace{2in}
    \textmd{\textbf{\hmwkClass:\ \hmwkTitle}}\\
    \normalsize\vspace{0.1in}\small{Due\ on\ \hmwkDueDate\ }\\
    \vspace{3in}
}

\author{\textbf{\hmwkAuthorName}}
\date{}

\renewcommand{\part}[1]{\textbf{\large Part \Alph{partCounter}}\stepcounter{partCounter}\\}

%
% Various Helper Commands
%

% Useful for algorithms
\newcommand{\alg}[1]{\textsc{\bfseries \footnotesize #1}}

% For derivatives
\newcommand{\deriv}[1]{\frac{\mathrm{d}}{\mathrm{d}x} (#1)}

% For partial derivatives
\newcommand{\pd}[2]{\frac{\partial}{\partial #1} (#2)}

\newcommand{\pdd}[2]{\frac{\partial #1}{\partial #2}}

% Integral dx
\newcommand{\dx}{\mathrm{d}x}

% Alias for the Solution section header
\newcommand{\solution}{\textbf{\large Solution}}

% Probability commands: Expectation, Variance, Covariance, Bias
\newcommand{\E}{\mathrm{E}}
\newcommand{\Var}{\mathrm{Var}}
\newcommand{\Cov}{\mathrm{Cov}}
\newcommand{\Bias}{\mathrm{Bias}}

% Some useful macros
\input{./macros.tex}

\begin{document}

\maketitle

\pagebreak

% Problem 1
\begin{homeworkProblem}
The proof of convergence of 1-step methods in Section 6.3 shows that the 
global error goes to zero as $k\goto 0$.
However, this bound may be totally useless in estimating the actual error
for a practical calculation.

For example, suppose we solve $u'=-10u$ with $u(0)=1$ up to time $T=10$, so
the true solution is $u(T)=e^{-100} \approx 3.7\times 10^{-44}$.
Using forward Euler with a time step $k=0.01$,  
the computed solution is $U^N = (.9)^{1000}\approx 1.75 \times
10^{-46}$, and so $E^N \approx 3.7 \times 10^{-44}$.
Since $L=10$ for this problem, the error bound (6.16) gives
\begin{equation}\label{eq:1}
\|E^N\| \leq e^{100}\cdot 10 \cdot \|\tau \|_\infty \approx 2.7 \times
10^{44} \|\tau\|_\infty.
\end{equation} 
Here $\|\tau\|_\infty =|\tau^0| \approx 50 k$, so this upper bound on the
error does go to zero as $k\goto 0$, but obviously it is not a realistic
estimate of the error. It is too large by a factor of about $10^{88}$.

The problem is that the estimate (6.16) is based on the Lipschitz
constant $L=|\lambda|$, which gives a bound that grows exponentially in time
even when the true and computed solutions are decaying exponentially.

\begin{enumerate}
\item Determine the computed solution and error bound (6.16) for the problem
$u' = 10u$ with $u(0)=1$ up to time $T=10$.  {\em Correction: Ignore the
following statement, since it's not correct...}  Note that the error bound is
the same as in the case above, but now it is a reasonable estimate of the
actual error.

\item A more realistic error bound for the case where $\lambda<0$ can be
obtained by rewriting (6.17) as
\[
\unp = \Phi(U^n)
\]
and then determining the Lipschitz constant for the function $\Phi$.
Call this constant $M$.  Prove that if $M\leq 1$  and $E^0=0$ then
\[
|E^n| \leq  T\|\tau\|_\infty
\]
for $nk\leq T$, a bound that is similar to (6.16) but without the
exponential term.


\item Show that for forward Euler applied to $u'=\lambda
u$ we can take $M = |1+k\lambda|$.  Determine $M$ for the case $\lambda =
-10$ and $k=0.01$ and use this in the bound from part (b).
Note that this is much better than the bound \eqn{eq:1}.
{\em But note that it's still not a very sharp bound.}

\end{enumerate}
    
    \vskip 1cm
    \textbf{Solution:} \\
    
    \begin{enumerate}
    
    % Part a
    \item We have 
    $$ \tau^n = \half ku''(t_n) = \half k(100u) =50ke^{10t_n} $$
    so 
    $$\|\tau\|_{\infty} = \half e^{100}.$$
    
    The solution computed by Forward Euler will be 
    \begin{align*}
        U^{n+1} &= U^n + k(10U^n) \\
         &= (1 + 10k)U^n \\
         &= 1.1U^n \\
         \Rightarrow U^{n} = (1.1)^n. 
    \end{align*}
    For $n=1000$ this gives $U^n = (1.1)^{1000} ~ \approx ~ 2.47 \times 10^{41}$. At $t=10$, the exact solution is
    $u(10) = e^{100} ~\approx ~2.69 \times 10^{43}$ which implies the absolute error in our approximation is 
    $\approx ~ 2.66 \times 10^{43}.$ Since the Lipschitz constant for this problem is $L = |\lambda| = 10$ 
    the error bound (\ref{eq:1}) gives 
    $$ |E^n| \leq e^{Lt_n} \cdot 10 \cdot \|\tau \|_\infty = e^{100}\cdot 5 \cdot e^{100} \approx 3.6 \times 10^{87}. $$
    
    % Part b
    \item
    Suppose $M\leq1$ is a Lipschitz constant for $\Phi$ and $E^0 =0.$ Then since
    $$ U^{n+1} = \Phi(U^n)$$
    and
    $$ u(t_{n+1}) = \Phi(u(t_n)) + k\tau^n, $$
    we have
    \begin{align*}
          E^{n+1} &= \Phi(U^n) - \Phi(u(t_n)) - k\tau^n \\
          \Rightarrow |E^{n+1}| &\leq |\Phi(U^n) - \Phi(u(t_n))| + |k\tau^n| \\
          &\leq M|E^n| + |k\tau^n| \\
          &= M|\Phi(U^{n-1}) - \Phi(u(t_{n-1})) - k\tau^{n-1}| + k|\tau^n| \\
          &\vdots \\
          \Rightarrow |E^n| &\leq M^n|E^0| + k \sum^{n}_{m=1}M^{n-m}|\tau^{m-1}| \\
          &= k \sum^{n}_{m=1}M^{n-m}|\tau^{m-1}| ~\leq k \sum^{n}_{m=1}|\tau^{m-1}| \\
          &\leq k \sum^{n}_{m=1}\|\tau\|_\infty ~\leq nk\|\tau\|_\infty \\
          &\leq T\|\tau\|_\infty
    \end{align*}
    for $nk\leq T$.
    
    % Part c
    \item
    The Forward Euler method for this problem is given by
    $$ U^{n+1} = U^n + k\Psi(U^n,t_n,k)$$
    where $\Psi(U^n,t_n,k) = f(U^n) = \lambda U^n.$ Hence
    $$ \Phi(U^n,t_n,k) = U^n + k\lambda U^n = U^n(1+k\lambda).$$
    Then for any $U^*$ we have
    \begin{align*}
        |\Phi(U^n,t_n,k) - \Phi(U^*,t_n,k)| &= |(1+k\lambda)U^n - (1+k\lambda)U^*| \\
        &\leq |1+k\lambda||U^n - U^*|
    \end{align*}
    so we can take $M = |1+k\lambda|$ as a Lipschitz constant for $\Phi.$ When $\lambda=-10$ and 
    $k=0.01$, $M=|1 - 0.1|=0.9$ Using the error bound from (b) (and observing that $E^0=0$ and $M<1$), this tells us that 
    $$ |E^n| \leq T\|\tau \|_\infty = 10\cdot \half e^{100} = 5e^{100} \approx 1.34\times10^{44}.$$
    
    \end{enumerate}
  
  
  
\end{homeworkProblem}

\pagebreak

% Problem 2
\begin{homeworkProblem}
   Which of the following Linear Multistep Methods are convergent?  For 
the ones that are not, are they inconsistent, or not zero-stable, or both?
 \begin{enumerate}
 \item $U^{n+2} = \half U^{n+1} + \half U^{n} + 2kf(U^{n+1})$
 \item $\unp = U^n$ 
 \item $U^{n+4} = U^{n} + \frac 4 3 k(f(U^{n+3})+f(U^{n+2})+f(U^{n+1}))$
 \item $U^{n+3} = -U^{n+2} + U^{n+1} +U^{n}+2k(f(U^{n+2})+f(U^{n+1}))$.
 \end{enumerate}
    
    \vskip 1cm
    \textbf{Solution:} \\
    A linear multistep method is zero-stable if its characteristic polynomial $\rho(\zeta)$ satisfies the root condition,
    is consistent if and only if $\rho(1)=0$ and $\rho '(1) = \sigma(1)$, where $\sigma(\zeta)$ is the other
    characteristic polynomial of the method, and is convergent if and only if it is both zero-stable and consistent.
    
    \begin{enumerate}
    
    % Part a
    \item 
        For this method
        \begin{align*}
            \rho(\zeta) &= \zeta^2 - \half \zeta - \half = (\zeta -1)(\zeta +\half ), \\
            \sigma(\zeta) &= 2\zeta.
        \end{align*}
        Then $\rho(1) = 0$ and $\rho ' (1) = 2(1) - \half = \frac{3}{2} \neq 2 = \sigma(1),$ so the method is
        inconsistent.
        However, the roots of $\rho$ are $-\half$ and $1$ so it is zero-stable. Overall the method is not convergent 
        since it is inconsistent.
    
    % Part b
    \item
        Here we have
        \begin{align*}
            \rho(\zeta) &= \zeta-1 \\
            \sigma(\zeta) &= 0.
        \end{align*}
        Hence $\rho(1) = 0$, but $\rho '(1) = 1 \neq 0 = \sigma(1)$ so the method is inconsistent. The only root of
        $\rho$ is $1$ so the root condition is satisfied and the method is zero-stable. In conclusion the method is 
        not convergent since it is inconsistent.
        
    % Part c
    \item
        In this case
        \begin{align*}
            \rho(\zeta) &= \zeta^4 - 1 = (\zeta^2 -1)(\zeta^2 +1), \\
            \sigma(\zeta) &= \frac{4}{3}(\zeta^3 + \zeta^2 + \zeta).
        \end{align*}
        $\rho(1) = 0$ and $\rho '(1) = 4(1)^3= 3\cdot \frac{4}{3} = \sigma(1)$ so the method is consistent. The roots of
        $\rho,~ \pm 1, \pm i$ satisfy the root condition, so the method is stable is well. Therefore this linear multistep
        method is convergent.
        
    % Part d
    \item
        This method has
        \begin{align*}
            \rho(\zeta) &= \zeta^3 + \zeta^2 - \zeta - 1 = (\zeta^2-1)(\zeta +1), \\
            \sigma(\zeta) &= 2\zeta^2 + 2\zeta.
        \end{align*}
        We have $\rho(1)=0$ and $\rho '(1) = 3(1)^2 + 2(1) - 1 = 4 = 2+2 = \sigma(1)$ so this method is consistent.
        However it is not zero-stable as $1$ is a double root which lies on the unit circle. Hence it is not convergent.
    \end{enumerate}
    
\end{homeworkProblem}

\pagebreak

% Problem 3
\begin{homeworkProblem}
    \begin{enumerate}
\item Determine the general solution to the linear difference equation
$2U^{n+3} - 5U^{n+2} + 4U^{n+1} - U^n = 0$.

{\bf Hint:} One root of the characteristic polynomial is at $\zeta=1$.

\item Determine the solution to this difference equation with the starting
values $U^0=11$, $U^1=5$, and $U^2=1$.  What is $U^{10}$?

\item Consider the LMM
\[
2U^{n+3} - 5U^{n+2} + 4U^{n+1} - U^n = k(\beta_0 f(U^n) + \beta_1 f(U^{n+1})).
\]
For what values of $\beta_0$ and $\beta_1$ is local truncation error
$\bigo(k^2)$?

\item Suppose you use the values of $\beta_0$ and $\beta_1$ just determined
in this LMM.  Is this a convergent method?

\end{enumerate} 

    \vskip 1cm
    \textbf{Solution:} \\
    
    \begin{enumerate}
    
    % Part a
    \item 
        This linear difference equation has the following characteristic polynomial,
        $$ \rho(\zeta) = 2\zeta^3 - 5\zeta^2+4\zeta-1 =(\zeta-1)(2\zeta^2-3\zeta+1)= (\zeta-1)^2(2\zeta-1). $$
        We have a repeated root so the general solution is of the form
        \begin{equation}\label{eq:2}
            U^n = c_1 + nc_2 + (\half)^nc_3.
        \end{equation}
    
    % Part b
    \item
        The initial data $U^0=11$, $U^1=5$, and $U^2=1$ gives us a system of equations which we may solve to determine
        $c_1, ~c_2$, and $c_3.$ Substituting the known values for $U^0,~U^1,$ and $U^2$ into (\ref{eq:2}), we get
        \begin{align*}
            c_1 &+ c_2 &+ c_3 &= 11 \\
            c_1 &+ c_2 &+ \half c_3 &=5 \\
            c_1 &+ 2c_2 &+ \frac{1}{4}c_3 &=1
        \end{align*}
        $$\Rightarrow c_1=3,c_2=-2,c_3=8.$$
        Thus the particular solution is given by
        \begin{align*}
            U^n &= 3 -2n + 8(\half)^n = 3 - 2n + 2^{3-n} \\
            \Rightarrow U^{10} &= 3 - 2(10) + 2^{-7} = \frac{1}{128} - 17.
        \end{align*}
        
    \pagebreak
        
    % Part c 
    \item   
        In order to construct a method which has $O(k^2)$ local truncation error, we need to force the $O(1)$ and $O(k)$
        terms of the local truncation error to vanish (the expansion is given in Section (5.9.1)). Note, the $O(k^{-1})$
        them is determined by the coefficients on the $U^j$ terms. Since $\rho(1) = 2-5+4-1=0$ this term is already
        taken care of. The $O(1)$ term is 
        $$u'(t_n)\sum^3_{j=0} (j\alpha_j-\beta_j),$$
        where $\alpha_j$ is the coefficient on $U^{n+j}$. Forcing this term to vanish is equivalent to enforcing
        $$ \beta_0 + \beta_1 = 4-(2\cdot5) + 6 = 0.$$
        The $O(k)$ term is
        $$ u''(t_n)\sum^3_{j=0}(\half j^2\alpha_j - j\beta_j).$$
        Setting this equal to $0$ gives the condition
        $$\beta_1 = 1.$$
        Solving these two equations yields $\beta_0 = -1, ~\beta_1=1$;
    
    % Part d
    \item
        In (a) we saw $\rho(\zeta) = (\zeta-1)^2(2\zeta-1)$ which has a double root of 1, which lies on the unit circle. 
        The root condition is not satisfied so the method is not zero-stable. This can also be seen from the general
        solution derived in (b). Thus the method is not convergent (although it is consistent).
    
    \end{enumerate}
    

\end{homeworkProblem}


% Problem 4
\begin{homeworkProblem}
    Consider the midpoint method $\unp = \unm + 2kf(U^n)$ applied to the test
problem $u' = \lambda u$.  The method is zero-stable and second order
accurate, and hence convergent.  If $\lambda<0$ then the true solution
is exponentially decaying.

On the other hand, for $\lambda<0$ and $k>0$ the point $z=k\lambda$ is never
in the region of absolute stability of this method (see Example 7.7),
and hence it seems that the numerical solution should be growing
exponentially for any nonzero time step.  (And yet it converges to
a function that is exponentially decaying.)

\begin{enumerate} 
\item
Suppose we take $U^0=\eta$, use Forward Euler to generate $U^1$, and then
use the midpoint method for $n=2,~3,~\ldots$.  Work out the exact solution
$U^n$ by solving the linear difference equation and explain how the apparent
paradox described above is resolved.

\item Devise some numerical experiments to illustrate the resolution of the
paradox.
\end{enumerate}

    \vspace{1cm}
    \textbf{Solution:} \\ 
    
    \begin{enumerate}
    
    % Part a
    \item 
        When applied to the test problem $u'=\lambda u$ the midpoint method can be written
        $$U^{n+1} = U^{n-1} + 2k\lambda U^n \Rightarrow U^{n+2} - 2k\lambda U^{n+1} - U^n = 0.$$
        The characteristic polynomial is then
        $$\rho(\zeta) = \zeta^2 - 2k\lambda\zeta - 1$$
        which has the following roots
        $$\zeta_{1,2} = k\lambda \pm \sqrt{(k\lambda)^2+1}.$$
        If we use one iteration of Forward Euler to obtain $U^1$ with an (exact) initial guess of $U^0=\eta$, then
        $$U^1 = \eta + k\lambda \eta = \eta(1+k\lambda).$$
        The general solution of the midpoint method is
        $$U^n = c_1(\zeta_1)^n + c_2(\zeta_2)^n,$$
        so we can determine $c_1$ and $c_2$ using the particular values of $U^0$ and $U^1$ we are working with. To this end,
        \begin{align*}
            U^0 &= \eta = c_1 + c_2 \\
            &\Rightarrow c_1 = \eta-c_2 \\
            U^1 &= (k\lambda + \sqrt{(k\lambda)^2+1})c_1 + (k\lambda - \sqrt{(k\lambda)^2+1})c_2 \\
            &\vdots\\
            &\Rightarrow c_1 = \frac{\eta}{2}\cdot \frac{1+\sqrt{1+(k\lambda)^2}}{\sqrt{1+(k\lambda)^2}}, 
            ~c_2 = \frac{\eta}{2}\cdot \frac{-1+\sqrt{1+(k\lambda)^2}}{\sqrt{1+(k\lambda)^2}}.
        \end{align*}
        
        Thus
            \[
            U^n = \frac{\eta}{2}\cdot \frac{1+\sqrt{1+(k\lambda)^2}}{\sqrt{1+(k\lambda)^2}}\cdot 
             (k\lambda + \sqrt{(k\lambda)^2+1})^n + 
            \frac{\eta}{2}\cdot \frac{-1+\sqrt{1+(k\lambda)^2}}{\sqrt{1+(k\lambda)^2}} \cdot
            (k\lambda - \sqrt{(k\lambda)^2+1})^n.
            \]
        If $\lambda<0$ and $k>0$ then second root satisfies,
        $|\zeta_2 = k\lambda - \sqrt{(k\lambda)^2+1}| > 1.$ Normally this would indicate that the method should be
        unstable on the problem at
        hand, since $\zeta_2$ is exponentiated at each iteration. However further analysis reveals that for $k$ small enough,
        this is not the case. First observe that $c_2$ is $O(k^2)$ for $k$ small since
        \begin{align*}
            c_2 &= \frac{\eta}{2}\cdot \frac{-1+\sqrt{1+(k\lambda)^2}}{\sqrt{1+(k\lambda)^2}} = \frac{\eta}{2}\cdot 
            \frac{-1 + (1 + \half(k\lambda)^2 + O((k\lambda)^4)}{\sqrt{1+(k\lambda)^2}} \\
            &= \frac{\eta}{2}\cdot \frac{\half(k\lambda)^2 + O((k\lambda)^4}{\sqrt{1+(k\lambda)^2}} =  
            \frac{\eta}{2}\cdot(\half(k\lambda)^2 + O((k\lambda)^4) \cdot (1 - \half(k\lambda)^2 + O((k\lambda)^4)) \\
            &= O(k^2),
        \end{align*}
        where we have used the Taylor expansions for $(1+x^2)^\half$ and $(1+x^2)^{-\half}$ for small $|x|.$
        Now we turn our attention to the root $\zeta_2.$ Suppose we are interested in computing an approximation to the test
        problem at a fixed time $T = nk.$ Again making use of a Taylor expansion, we have
        \begin{align*}
            \zeta_2 &= k\lambda - \sqrt{1+(k\lambda)^2} = k\lambda - (1 + \half(k\lambda)^2 + O((k\lambda)^2) \\
            &= -1 + k\lambda + O(k^2) \\
            &\Rightarrow \zeta_2^n = (-1+k\lambda + O(k^2))^n \\
            &=(-1)^n(1 + |\lambda|k + O(k^2))^n \\
            &=(-1)^n(1+ \frac{|\lambda|T}{n} + O(k^2))^n \approx (-1)^n(1+ \frac{|\lambda|T}{n})^n.
        \end{align*}
        As $k\goto0,$ since $T=kn$ is fixed, $n\goto\infty$ and $(-1)^n(1+ \frac{|\lambda|T}{n})^n \approx (-1)^ne^{|\lambda|T}.$
        This implies that $\zeta_2^n$ is bounded as $n$ gets large and $k$ small. Since $c_2=O(k^2)$, for small $k$
        (and large $n$), we have
        \begin{align*}
            c_2\zeta_2^n &=  \frac{\eta}{2}\cdot \frac{-1+\sqrt{1+(k\lambda)^2}}{\sqrt{1+(k\lambda)^2}}\cdot 
            (k\lambda - \sqrt{1+(k\lambda)^2})^n
             = O(k^2) \cdot (k\lambda - \sqrt{1+(k\lambda)^2})^n \\
             &\approx O(k^2)e^{|\lambda|T} = O(k^2).
        \end{align*}
        Hence the term which had the potential to disrupt our approximation by blowing up is actually very small for small $k$.
        The other root is less than $1$ in modulus, so it causes no instability, so the midpoint method works on the test problem.
    % Part b
    \item
        See the attached Juliabox notebook.
    
    \end{enumerate}
   
    
\end{homeworkProblem}


% Problem 5
\begin{homeworkProblem}
    Perform numerical experiments to confirm the claim made in Example 7.10.
    
    \vspace{1cm}
    \textbf{Solution:}\\
    See the attached Juliabox notebook.
    
\end{homeworkProblem}



\end{document}

