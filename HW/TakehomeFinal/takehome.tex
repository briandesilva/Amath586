\documentclass{article}

\usepackage{fancyhdr}
\usepackage{extramarks}
\usepackage{amsmath}
\usepackage{amsthm}
\usepackage{amsfonts}
\usepackage{amssymb}
\usepackage{graphicx}
\usepackage{caption,subcaption}
\usepackage{subfig}

%
% Basic Document Settings
%

\topmargin=-0.45in
\evensidemargin=0in
\oddsidemargin=0in
\textwidth=6.5in
\textheight=9.0in
\headsep=0.25in

\linespread{1.1}

\pagestyle{fancy}
\lhead{\hmwkAuthorName}
\chead{\hmwkClass:\ \hmwkTitle}
\rhead{\firstxmark}
\lfoot{\lastxmark}
\cfoot{\thepage}

\renewcommand\headrulewidth{0.4pt}
\renewcommand\footrulewidth{0.4pt}

\setlength\parindent{0pt}

%
% Create Problem Sections
%

\newcommand{\enterProblemHeader}[1]{
    \nobreak\extramarks{}{Problem \arabic{#1} continued on next page\ldots}\nobreak{}
    \nobreak\extramarks{Problem \arabic{#1} (continued)}{Problem \arabic{#1} continued on next page\ldots}\nobreak{}
}

\newcommand{\exitProblemHeader}[1]{
    \nobreak\extramarks{Problem \arabic{#1} (continued)}{Problem \arabic{#1} continued on next page\ldots}\nobreak{}
    \stepcounter{#1}
    \nobreak\extramarks{Problem \arabic{#1}}{}\nobreak{}
}

\setcounter{secnumdepth}{0}
\newcounter{partCounter}
\newcounter{homeworkProblemCounter}
\setcounter{homeworkProblemCounter}{1}
\nobreak\extramarks{Problem \arabic{homeworkProblemCounter}}{}\nobreak{}

%
% Homework Problem Environment
%
% This environment takes an optional argument. When given, it will adjust the
% problem counter. This is useful for when the problems given for your
% assignment aren't sequential. See the last 3 problems of this template for an
% example.
%
\newenvironment{homeworkProblem}[1][-1]{
    \ifnum#1>0
        \setcounter{homeworkProblemCounter}{#1}
    \fi
    \section{Problem \arabic{homeworkProblemCounter}}
    \setcounter{partCounter}{1}
    \enterProblemHeader{homeworkProblemCounter}
}{
    \exitProblemHeader{homeworkProblemCounter}
}

%
% Homework Details
%   - Title
%   - Due date
%   - Class
%   - Section/Time
%   - Instructor
%   - Author
%

\newcommand{\hmwkTitle}{Take-home Final}
\newcommand{\hmwkDueDate}{June 11, 2015}
\newcommand{\hmwkClass}{Amath 586}
\newcommand{\hmwkAuthorName}{Brian de Silva}

%
% Title Page
%

\title{
    \vspace{2in}
    \textmd{\textbf{\hmwkClass:\ \hmwkTitle}}\\
    \normalsize\vspace{0.1in}\small{Due\ on\ \hmwkDueDate\ }\\
    \vspace{3in}
}

\author{\textbf{\hmwkAuthorName}}
\date{}

\renewcommand{\part}[1]{\textbf{\large Part \Alph{partCounter}}\stepcounter{partCounter}\\}

%
% Various Helper Commands
%

% Useful for algorithms
\newcommand{\alg}[1]{\textsc{\bfseries \footnotesize #1}}

% For derivatives
\newcommand{\deriv}[1]{\frac{\mathrm{d}}{\mathrm{d}x} (#1)}

% For partial derivatives
\newcommand{\pd}[2]{\frac{\partial}{\partial #1} (#2)}

\newcommand{\pdd}[2]{\frac{\partial #1}{\partial #2}}

% Integral dx
\newcommand{\dx}{\mathrm{d}x}

% Alias for the Solution section header
\newcommand{\solution}{\textbf{\large Solution}}

% Probability commands: Expectation, Variance, Covariance, Bias
\newcommand{\E}{\mathrm{E}}
\newcommand{\Var}{\mathrm{Var}}
\newcommand{\Cov}{\mathrm{Cov}}
\newcommand{\Bias}{\mathrm{Bias}}

% Some useful macros
\input{./macros.tex}

\begin{document}

\maketitle

\pagebreak

% Problem 1
\begin{homeworkProblem}
    Consider the method
\begin{equation} \label{a}
U_j^{n+1} = U_j^n - \frac{ak}{2h}(U_j^n-U_{j-1}^n + U_j^{n+1}-U_{j-1}^{n+1}).
\end{equation}
for the advection equation $u_t+au_x=0$ on $0\leq x \leq 1$ with periodic
boundary conditions, on a grid with spacing $h= 1 / (m+1)$ for some integer
$m$.  

\begin{enumerate}
\item This method can be viewed as the trapezoidal method applied to an ODE
system $U'(t) = AU(t)$ arising from a method of lines discretization of the
advection equation.  What is the matrix $A$?  Don't forget the boundary
conditions.

\item Determine the eigenvalues of the matrix $A$.  {\bf Hint:} the
eigenvectors are the same as for similar matrices you have seen.

\item Suppose we want to fix the Courant number $ak/h$ as $k,~h\goto 0$.
For what range of Courant numbers will the method be stable if $a>0$?
If $a<0$?  Justify your answers in terms of eigenvalues of the matrix $A$
and  the stability region of the trapezoidal method.

\item For what range of $ak/h$ will the CFL condition be satisfied for this
method (with periodic boundary conditions)?  Consider both $a>0$ and $a<0$.
Is satisfying the CFL condition sufficient for stability in all cases?

\end{enumerate}

    
    \vskip 1cm
    \textbf{Solution:} \\
    
\begin{enumerate}

% Part (a)
\item Consider the following ODE system:
\[
U'(t) = -\frac{a}{2h}(U_j(t)-U_{j-1}(t)).
\]
If we apply the trapezoidal method to this system we obtain (\ref{a}). With periodic boundary conditions, this
suggests that we can write this system as $U'(t) = AU(t)$ with $A$ given by
\[
A = -\frac{a}{h}\bcm 1&&&&&-1 \\ -1&1&&&& \\ &-1&1&&& \\ &&\ddots&\ddots&& \\ &&&-1&1& \\ &&&&-1&1 \ecm.
\]

% Part (b)
\item It is easy to check that if $\textbf{u}^p = (e^{2\pi iph},e^{4\pi iph},\dots,e^{2(m+1)\pi iph})^T$ then
\[
A\textbf{u}^p = -\frac{a}{h}\left( \frac{e^{2\pi iph}-1}{e^{2\pi iph}} \right) \textbf{u}^p, \qquad p=1,2,\dots, m+1.
\]
Thus $\textbf{u}^p$ are the eigenvalues of $A$ with eigenvalues $\lambda_p = 
-\frac{a}{h}\left(\frac{e^{2\pi iph}-1}{e^{2\pi iph}} \right),p=1,2,\dots,m+1$.
We can rewrite this expression in what is perhaps a more convenient form:
$$ \lambda_p = -\frac{a}{h}(1-e^{-2\pi iph})=-\frac{a}{h}(1-\text{cos}(2\pi ph)+i\text{sin}(2\pi ph)),\qquad p=1,2,\dots,m+1.$$

% Part (c)
\item Let $c=ak/h$, the Courant number. Recall that for the trapezoidal method the expression dictating stability
is given by
\[
R(z) = \frac{1+\half z}{1-\half z}.
\]
For a fixed value of $k$ the method is stable in the above problem if $|R(k\lambda^p)|\leq 1$ for every 
eigenvalue $\lambda^p$, and is unstable otherwise. Therefore we need $|1+\half k\lambda^p|\leq|1-\half k\lambda^p|$
in order for the method to be stable. Substituting in our expression for $\lambda^p$ gives the following
condition for stability:
\begin{align*}
~&~|1+c(1-\text{cos}(2 \pi ph) + i\text{sin}(2 \pi ph))/2| ~\leq~ |1-c(1-\text{cos}(2 \pi ph) + i\text{sin}(2 \pi ph))/2| \\
\Longrightarrow &~(1+c(1-\text{cos}(2 \pi ph))/2)^2 ~\leq~ (1-c(1-\text{cos}(2 \pi ph))/2)^2 \\
\Longrightarrow &~|1+c(1-\text{cos}(2 \pi ph))/2)| ~\leq~ |1-c(1-\text{cos}(2 \pi ph))/2|, \qquad p=1,2,\dots,m+1.
\end{align*}
Notice that $1-\text{cos}(2\pi ph) \leq 1$ for each $p=1,2,\dots, m+1$. Hence if $c>0$ (i.e. $a>0$), 
$c(1-\text{cos}(2 \pi ph))/2 \leq 0$ so the stability condition is satisfied for any positive value of $k$. If $c<0$
(equivalently $a<0$) then $c(1-\text{cos}(2 \pi ph))/2 \geq 0$ so the stability condition cannot be satisfied. We
conclude that the method is stable for all $c>0$ and unstable for all $c<0$.

% Part (d)
\item Using method (\ref{a}), the approximation $U^{n+1}_j$ depends on the points $U^{n+1}_j,~U^{n+1}_{j-1},~
U^n_j,$ and $U^n_{j-1}$. Because of the dependence on points at time $t_{n+1}$ (i.e. because the method is
explicit) as the grid is refined the numerical domain of dependence of a point $(X,T)$ approaches 
$(-\infty,X]$. Since we have periodic boundary conditions on $[0,1]$ in this case, the numerical domain
of dependence converges to the entire interval $[0,1]$ as $k,h\goto0$. The domain of dependence of the PDE is
clearly always contained in the numerical one. Therefore the CFL condition is always satisfied, regardless of the
sign of $a$. So knowing that the CFL condition is satisfied does not actually provide us with any information.

If $a>0$ the method is stable, as we saw in part (c), and the CFL condition is automatically satisfied, so in this
case satisfying the CFL condition is sufficient for stability. For $a<0$ the CFL condition is satisfied, but the
method is unstable, so satisfying the CFL condition is insufficient for stability.

\end{enumerate}


\end{homeworkProblem}


% Problem 2
\begin{homeworkProblem}
    Consider the dispersive PDE $u_t + \gamma u_{xxx} = 0$ with periodic boundary
conditions on an interval $a \leq x \leq b$.  

\begin{enumerate} 
\item The third-derivative term could be approximated by either 
\begin{equation} \label{2a}
u_{xxx}(x_j,t_n) \approx \frac 1 {h^3}(-U_{j-2}^n + 3U_{j-1}^n - 3U_j^n + U_{j+1}^n)
\end{equation} 
or by 
\begin{equation} \label{2b}
u_{xxx}(x_j,t_n) \approx \frac 1 {h^3}(-U_{j-1}^n + 3U_{j}^n - 3U_{j+1}^n + U_{j+2}^n)
\end{equation} 
Show that either one is first-order accurate.  

\item Suppose you plan to use a trapezoid discretization in time for a
problem where $\gamma >0$.  Which form of spatial discretization
\eqn{2a} or \eqn{2b} would be better?  Justify your answer.

\item Implement a Crank-Nicolson type method for this equation, which uses
the trapezoid method in time together with the spatial discretization from
above that you determined is best for the case $\gamma >0$.
Your implementation should allow arbitrary values of the endpoints $a$ and
$b$ of the spatial interval, since two different choices are required below.

Test your method by showing it is first order accurate 
for initial data $u(x,0) = \\text{sin}(px)$ for integer $p$ such as $p=2$ on the
interval $0 \leq x \leq 2\pi$.  Note that 
you can compute the exact solution for such data.
\end{enumerate}
    
    
    \vskip 1cm
    \textbf{Solution:} \\
    
\begin{enumerate}

% Part (a)
 \item Let $u$ denote $u(x,t)$. We use the following Taylor expansions:
 \begin{align*}
 u(x+2h,t) &= u+2hu_x +2h^2u_{xx}+\frac{4h^3}{3}u_{xxx}+\frac{16h^4}{24}u_{4x} + O(h^5), \\
 u(x+h,t) &= u+hu_x + \frac{h^2}{2}u_{xx}+\frac{h^3}{6}u_{xxx}+\frac{h^4}{24}u_{4x}+O(h^5),\\
 u(x-h,t) &= u-hu_x + \frac{h^2}{2}u_{xx}-\frac{h^3}{6}u_{xxx}+\frac{h^4}{24}u_{4x}+O(h^5),\\
 u(x-2h,t) &= u-2hu_x +2h^2u_{xx}-\frac{4h^3}{3}u_{xxx}+\frac{16h^4}{24}u_{4x} + O(h^5).
 \end{align*}
 
 Substituting these into the right-hand side of (\ref{2a}) and grouping terms in of powers of $h$ gives
 \begin{align*}
 &\frac{1}{h^3}\left((3-3)u+h(1-3+2)u_x+h^2\left(\half+\frac{3}{2}-2\right)u_{xx}+
 h^3\left(\frac{1}{6}-\half+\frac{4}{3}\right)u_{xxx} +
 h^4\left(\frac{1}{24}+\frac{3}{24}-\frac{16}{24}\right)u_{4x}+O(h^5)\right) \\
 &= u_{xxx}-\frac{h}{2}u_{4x} +O(h^2).
 \end{align*}
 
 Carrying out the same procedure for the right-hand side of (\ref{2b}) yields
 \begin{align*}
 &\frac{1}{h^3}\left((3-3)u+h(2-3+1)u_x+h^2\left(2-\frac{3}{2}-\half\right)u_{xx}+
 h^3\left(\frac{4}{3}-\half+\frac{1}{6}\right)u_{xxx} +
 h^4\left(\frac{16}{24}-\frac{3}{24}-\frac{1}{24}\right)u_{4x}+O(h^5)\right) \\
 &= u_{xxx}+\frac{h}{2}u_{4x} +O(h^2).
 \end{align*}
 
 Hence both methods give first order approximations to $u_{xxx}$.
 
% Part(b)
 \item Notice that a method based on (\ref{2a}) draws more information from points on the left and (\ref{2b}) from
 points on the right. If $u(x,0) = u_0(x)$ then we have
 \begin{align*}
 ~&~u_t = -\gamma u_{xxx} \\
 \Longrightarrow &~ \hat u_t = \gamma i\xi^3\hat u \\
 \Longrightarrow &~\hat u~ = e^{i\gamma \xi^3t}\hat u_o.
 \end{align*}
 Taking the inverse Fourier transform gives
 \begin{equation*}
 u(x,t) = \frac{1}{\sqrt{2\pi}}\int^\infty_\infty \hat u_o(\xi)e^{i\xi(x+\gamma\xi^2t)}d\xi.
 \end{equation*}
 Hence the Fourier component corresponding to wavenumber $\xi$ moves with velocity $-\gamma \xi^2$. It follows
 that solutions of the PDE will move to the left if $\gamma >0$ and to the right if $\gamma<0$. Since the 
 trapezoidal method is absolutely stable and gives rise to a method which has a numerical domain of dependence
 including the entire interval $[a,b]$, we do not need to worry as much about stability and the CFL condition
 being satisfied when choosing between (\ref{2a}) and (\ref{2b}). Instead we should choose the scheme better
 suited to the PDE. In this case a numerical method using (\ref{2b}) is preferable since it samples more points
 to the right, which is consistent with our expectation that information will be travelling to the left as time
 evolves.
 
 \pagebreak
 
% Part (c)
 \item Assuming a solution of the form $u(x,t) = \text{sin}(px + a\gamma t)$ and substituting into the PDE to solve for
 $a$ gives $u(x,t) = \text{sin}(px + p^3\gamma t)$ as an exact solution.
 
 \begin{figure}
 \centering
 \includegraphics[width=0.6\textwidth]{2cgood}
 \caption{Error at $t=1$ as $h$ and $k$ are refined for a method based on (\ref{2b})}
 \label{fig:2cgood}
 \end{figure}
 We fix $p=2$ and $\gamma  =1$ and approximate the solution from $t=0$ to $t=1$ on the interval 
 $0\leq x\leq 2\pi$. Since (\ref{2b}) is first order accurate and the trapezoidal method is second order accurate
 it is reasonable to expect the method obtained by combining the two is $O(k^2+h)$ accurate. Hence we fix
 $k = \sqrt{h}/2$ as we refine the grid to create the error plot in Figure (\ref{fig:2cgood}).
 
 The numerically approximated order of accuracy is $0.93$. If we relax the condition on $k$ and allow $k=O(h)$
 then the accuracy improves substantially. By modifying the relationship between $h$ and $k$ we obtain approximate
 orders of accuracy ranging from $0.88$ to $1.63$.
 
 \begin{figure}
 \centering
 \includegraphics[width=0.6\textwidth]{2cbad}
 \caption{Error at $t=1$ as $h$ and $k$ are refined for a method based on (\ref{2a})}
 \label{fig:2cbad}
 \end{figure}
 
 Just to check that we were correct in choosing to base our method off (\ref{2b}) rather than (\ref{2a}) we have
 computed the error using all the same parameters, but using the alternate method. The results are presented in
 Figure (\ref{fig:2cbad}).
 
 \vskip 1cm
 The code used to run the numerical trials can be found in the appendix.
 
\end{enumerate}
    
    
\end{homeworkProblem}

\pagebreak

% Problem 3
\begin{homeworkProblem}
    Now consider the KdV equation $u_t + 6uu_x + u_{xxx} = 0$.  
This Korteweg - de Vries equation is famous --- read about it if you are not
familiar with it.  Even though this equation is nonlinear,
there are exact ``solitary wave'' solutions that consist of a translating wave of
the form $u(x,t) = \eta(x-\alpha^2 t)$ for special initial data of the form
\begin{equation} \label{3a}
\eta(x) = \frac {\alpha^2}{2} \text{sech}^2\left(\alpha x/2\right)
\end{equation} 
for any choice of parameter $\alpha >0$.  Note that taller pulses
propagate more rapidly.  An amazing feature of such problems is the
way two such pulses interact as ``solitons''. 

Construct a first-order accurate method by combining your solution to Problem 2
with a first-order upwind method for the hyperbolic equation
\begin{equation} \label{3b}
u_t + 3(u^2)_x = 0.
\end{equation} 
Use the first-order fractional step approach to combine the methods --- i.e.
in each time step, first advance the solution by taking a step with the $u_t
+ u_{xxx} = 0$ to get an intermediate solution $U^*$, and then use this as
initial data for a step in which you solve \eqn{3b} by the upwind method for
this nonlinear conservation law.  

Again use periodic boundary conditions.  The true solution is not periodic,
but decays exponentially to zero so as long as the interval is long enough
relative to the width of the solitary wave, so this should not affect errors
computed using a periodic extension of \eqn{3a}.

Test your method on the interval $-1 \leq x \leq 3$ with $\alpha=10$ out to
time $t=0.02$.  Demonstrate that your method is stable and first-order
accurate by refining the grid with $r = k/h$ fixed.

Estimate the maximum stable value of $r$ based on stability theory,
justifying your estimate, and confirm that this is consistent with
what you observe from your numerical method.

\vskip 10pt
Note that this is not a very accurate method!  If you want to play around
with soliton interaction (optional) you might try the following data:
\begin{equation} 
\eta(x) = \frac {\alpha_1^2}{2} \text{sech}^2\left(\alpha_1 (x + 2)/2
\right) + \frac {\alpha_2^2}{2} \text{sech}^2\left(\alpha_2 (x + 1)/2\right)
\end{equation} 
with $\alpha_1 = 25, ~\alpha_2= 16$ on the interval $-3\leq x \leq 2$.  
    
    \vskip 1cm
    \textbf{Solution:} \\
    It is stated in the text that if one can show $\| \mathcal{N}_A(U,k)\| \leq \|U\|$ and
    $\| \mathcal{N}_B(U,k)\| \leq \|U\|$ (where the same norm is used in both instances)
    then one can demonstrate the stability of a fractional step method using $\mathcal{N}_A$ and $\mathcal{N}_B$
    as its first and second steps. It also alerted us to the possibility that the stability of the two methods
    independently may not be enough to guarantee a fractional step method which is stable.
    However the task of proving the norm condition proved to be too daunting for the author, so keeping the 
    warning in mind, we cautiously attempt to see whether enforcing the separate stability conditions
    from each method can inform our numerical tests to determine $r = k/h$.
    
    
    As we argued in problem 2 we are not very worried about the stability of our solution to problem 2 since 
    it always satisfies the CFL condition and has very good stability properties due its use of the trapezoidal
    method, at least for $\gamma >0$. In this particular instance it is used in a fractional step method to
    solve the equation $u_t + u_{xxx}=0$, in which case $\gamma = 1 > 0$. Hence we expect it to behave well.
    
    We saw in class that the stability of the upwind method agrees with the restriction imposed by the CFL
    condition. Thus if we can determine the CFL condition for the hyperbolic equation being solved by the upwind
    method at each step, $u_t +3(u^2)_x=0$, then we will be able to get a rough estimate of the maximal value
    $r$ can take. First note that the PDE can be rewritten
    \[
    u_t + 6uu_x=0
    \]
    which we recognize as Burger's equation. This simple quasilinear PDE can be solved using the method of
    characteristics. If we use $\eta(x)$ for the initial data we obtain
    \[
    u(x,t) = \eta(\xi)
    \]
    where $\xi$ is given implicitly by 
    \[
    \xi = x -3\alpha^2\text{sech}^2(\alpha \xi/2)t.
    \]
    From the form of the solution we can see that the initial data is simply propagated to the right at varying
    speeds determined by $3\alpha^2\text{sech}^2(\alpha \xi/2)$. Since $0\leq \text{sech}(x)\leq 1$ 
    we can infer that the fastest moving data will move to the right with speed $3\alpha^2 = 300$ 
    (for $\alpha=10$). In fact this occurs for the initial data corresponding to $\xi=0$. Therefore, 
    at worst, the solution at a point $(X,T)$ has domain of dependence ${X-300T}$. For other values of 
    $\xi$ the domain of dependence of a point $(X,T)$ will be
    contained in the interval $[X-300T,X]$. Since the numerical domain of dependence fills in the interval
    $[X-T/r, X]$ as the grid is refined, we require $300\leq 1/r$ or $r\leq 1/300$.
    
    \begin{figure}
    \centering
    \includegraphics[width=0.6\textwidth]{3all}
    \caption{Error at $t=0.02$ as $h$ and $k$ are refined with $k=300h$}
    \label{fig:3}
    \end{figure}
    This restriction on $r$ agrees well with the numerical test. As we see in Figure (\ref{fig:3}) for $r=1/300$
    the method exhibits first order accuracy (approximate order of accuracy: 0.87). If we leave off the 
    two data points corresponding to the coarsest grids the approximate order of accuracy increases to $0.91$.
    If we allow $r$ to be slightly larger, say $1/275$, the method appears to be stable for smaller numbers
    of time steps, but as the grid is refined it eventually becomes unstable and the approximate solution blows
    up. It was difficult for me to test numerically exactly where the threshold lies due to
    the time needed to solve the problem on very fine grids. The largest value of $r$ for which I numerically
    verified instability was $r=1/277$. The method remained stable even for large numbers of time steps 
    when $r$ was taken to be $1/300$. Note that the initial data fed into the upwind method is actually the
    result of applying the method devised in problem 2 to the actual initial data $\eta(x)$, so we do not
    expect the analytic upper bound on $r$ to exactly match the numerical one. In any case, the numerical 
    results agree with the restriction found above.
    
    \vskip 1cm
    The code used to run the numerical trials can be found in the appendix.
    
\end{homeworkProblem}


        

\pagebreak

\hfill

\pagebreak
~~
\vskip 10cm
\centering
\huge \textbf{Appendix: Code}


\end{document}

