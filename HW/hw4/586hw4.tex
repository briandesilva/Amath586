\documentclass{article}

\usepackage{fancyhdr}
\usepackage{extramarks}
\usepackage{amsmath}
\usepackage{amsthm}
\usepackage{amsfonts}
\usepackage{amssymb}
\usepackage{graphicx}
\usepackage{caption,subcaption}
\usepackage{subfig}

%
% Basic Document Settings
%

\topmargin=-0.45in
\evensidemargin=0in
\oddsidemargin=0in
\textwidth=6.5in
\textheight=9.0in
\headsep=0.25in

\linespread{1.1}

\pagestyle{fancy}
\lhead{\hmwkAuthorName}
\chead{\hmwkClass:\ \hmwkTitle}
\rhead{\firstxmark}
\lfoot{\lastxmark}
\cfoot{\thepage}

\renewcommand\headrulewidth{0.4pt}
\renewcommand\footrulewidth{0.4pt}

\setlength\parindent{0pt}

%
% Create Problem Sections
%

\newcommand{\enterProblemHeader}[1]{
    \nobreak\extramarks{}{Problem \arabic{#1} continued on next page\ldots}\nobreak{}
    \nobreak\extramarks{Problem \arabic{#1} (continued)}{Problem \arabic{#1} continued on next page\ldots}\nobreak{}
}

\newcommand{\exitProblemHeader}[1]{
    \nobreak\extramarks{Problem \arabic{#1} (continued)}{Problem \arabic{#1} continued on next page\ldots}\nobreak{}
    \stepcounter{#1}
    \nobreak\extramarks{Problem \arabic{#1}}{}\nobreak{}
}

\setcounter{secnumdepth}{0}
\newcounter{partCounter}
\newcounter{homeworkProblemCounter}
\setcounter{homeworkProblemCounter}{1}
\nobreak\extramarks{Problem \arabic{homeworkProblemCounter}}{}\nobreak{}

%
% Homework Problem Environment
%
% This environment takes an optional argument. When given, it will adjust the
% problem counter. This is useful for when the problems given for your
% assignment aren't sequential. See the last 3 problems of this template for an
% example.
%
\newenvironment{homeworkProblem}[1][-1]{
    \ifnum#1>0
        \setcounter{homeworkProblemCounter}{#1}
    \fi
    \section{Problem \arabic{homeworkProblemCounter}}
    \setcounter{partCounter}{1}
    \enterProblemHeader{homeworkProblemCounter}
}{
    \exitProblemHeader{homeworkProblemCounter}
}

%
% Homework Details
%   - Title
%   - Due date
%   - Class
%   - Section/Time
%   - Instructor
%   - Author
%

\newcommand{\hmwkTitle}{Homework 4}
\newcommand{\hmwkDueDate}{May 12, 2015}
\newcommand{\hmwkClass}{Amath 586}
\newcommand{\hmwkAuthorName}{Brian de Silva}

%
% Title Page
%

\title{
    \vspace{2in}
    \textmd{\textbf{\hmwkClass:\ \hmwkTitle}}\\
    \normalsize\vspace{0.1in}\small{Due\ on\ \hmwkDueDate\ }\\
    \vspace{3in}
}

\author{\textbf{\hmwkAuthorName}}
\date{}

\renewcommand{\part}[1]{\textbf{\large Part \Alph{partCounter}}\stepcounter{partCounter}\\}

%
% Various Helper Commands
%

% Useful for algorithms
\newcommand{\alg}[1]{\textsc{\bfseries \footnotesize #1}}

% For derivatives
\newcommand{\deriv}[1]{\frac{\mathrm{d}}{\mathrm{d}x} (#1)}

% For partial derivatives
\newcommand{\pd}[2]{\frac{\partial}{\partial #1} (#2)}

\newcommand{\pdd}[2]{\frac{\partial #1}{\partial #2}}

% Integral dx
\newcommand{\dx}{\mathrm{d}x}

% Alias for the Solution section header
\newcommand{\solution}{\textbf{\large Solution}}

% Probability commands: Expectation, Variance, Covariance, Bias
\newcommand{\E}{\mathrm{E}}
\newcommand{\Var}{\mathrm{Var}}
\newcommand{\Cov}{\mathrm{Cov}}
\newcommand{\Bias}{\mathrm{Bias}}

% Some useful macros
\input{./macros.tex}

\begin{document}

\maketitle

\pagebreak

% Problem 1
\begin{homeworkProblem}
    Consider the following method for solving the heat equation
    $u_t=u_{xx}$:
    \[
    U_i^{n+2} = U_i^n + \frac{2k}{h^2}(U_{i-1}^{n+1} - 2U_i^{n+1} +
    U_{i+1}^{n+1}).
    \]
    \begin{enumerate}
    \item Determine the formal order of accuracy of this method 
    (in both space and time) based on computing the local truncation error.
    
    \item Suppose we take $k=\alpha h^2$ for some fixed $\alpha>0$ and refine
    the grid.  Show that this method fails to be 
    Lax-Richtmyer stable for any choice of $\alpha$.
    
    Do this in two ways: 
    \begin{itemize}
    \item Consider the MOL interpretation and the stability region of
    the time-discretization being used.
    \item Use von Neumann analysis and solve a quadratic equation for $g(\xi)$.
    \end{itemize} 
    
    \item What if we take $k=\alpha h^3$ for some fixed $\alpha>0$ and refine
    the grid. Would this method be stable?
    
    \end{enumerate}
    
    
    \vskip 1cm
    \textbf{Solution:} \\
    \begin{enumerate}
        
        % Part (a)
        \item Throughout this derivation of the local truncation error, let $u$ denote $u(x,t+k)$, $u_t$ $\pd{t}{u}|_{x,t+k}$, etc..
        Rather than expanding each term
        about $u(x,t)$, we expand about $u(x,t+k)$ and interpret the method as the midpoint rule applied at $u(x,t+k)$. We have the 
        following Taylor expansions:
        \begin{align*}
            u(x,t+2k) &= u + ku_t + \frac{k^2}{2}u_{tt} + \frac{k^3}{6}u_{ttt} +O(k^4), \\
            u(x,t) &= u - ku_t + \frac{k^2}{2}u_{tt} - \frac{k^3}{6}u_{ttt} +O(k^4), \\
            u(x+h,t) &= u + hu_x + \frac{h^2}{2}u_{xx} + \frac{h^3}{6}u_{xxx} + \frac{h^4}{4!}u_{xxxx} + O(h^5), \\
            u(x-h,t) &= u - hu_x + \frac{h^2}{2}u_{xx} - \frac{h^3}{6}u_{xxx} + \frac{h^4}{4!}u_{xxxx} + O(h^5).
        \end{align*}
        
        Substituting these into the local truncation error expression
        \[
        \tau(x,t+k) = \frac{u(x,t+2k) - u(x,t)}{2k} - \frac{1}{h^2}(u(x-h,t+k) - 2u(x,t+k) +u(x+h,t+k))
        \]
        and making obvious cancellations, we obtain
        \begin{align*}
        \tau(x,t+k)  &= u_t + \frac{k^2}{6}u_{tt}+O(k^3) - (u_{xx} + \frac{h^2}{12}u_{xxxx} + O(h^3)) \\
                     &= \left(\frac{k^2}{6} + \frac{h^2}{12}\right)u_{xxxx} + O(k^3 + h^3),
        \end{align*}
        where we have used $u_t = u_{xx}$ to cancel terms and replace $u_{tt} with u_{xxxx}$. Hence the method is formally
        second order accurate in both $h$ and $k$.
        
        \pagebreak
        
        % Part (b)
        \item \begin{enumerate}
            \item \textbf{Method of Lines}
            
            We interpret the above finite difference method as the midpoint method applied to the system of ODEs
            \[
            U_i'(t) = \frac{1}{h^2}(U_{i-1}(t) -2U_i(t)+U_{i+1}), \qquad i=1,2,\dots,m,
            \]
            which can be expressed more compactly as 
            \[
            U'(t) = AU(t) + g(t),
            \]
            where
            \[
            A = \frac{1}{h^2}\bcm -2 & 1 & & & & \\
            1 & -2 & 1 & & & \\
             & \ddots & \ddots & \ddots & & \\
             & & & & & \\
             & & & & & 1 \\
             & & & & 1 & -2
            \ecm
            \]
            and $g(t)$ simply takes care of the boundary terms.
            
            As was noted in class in order for this method to be stable we need each of the roots of $\pi(\zeta;k\lambda_p)$ to be less
            than one in magnitude, where $\lambda_p$ is the $p^{th}$ eigenvalue of $A$ and $\pi(\zeta;z)$ is defined with respect to
            the midpoint method. For this to hold we need $k\lambda_p \in S$ for each $p$, where $S$ is the stability region for the
            midpoint method. Or we at least need $k\lambda_p$ to approach $S$ as $k$ is refined.
            Recall that $S={z \in \complex| z = ib, ~ b\in \reals, ~ |b| \leq 1 }$. $A$ has eigenvalues 
            $\lambda_p = \frac{2}{h^2}(cos(p\pi h) - 1)$ for $p=1,2,\dots, m,$ so if the method is to be stable it must satisfy
            $k\lambda_p = \frac{2k}{h^2}(cos(p\pi h) - 1) \in S.$ The eigenvalue farthest from the origin is approximately
            $\frac{-4}{h^2}$, so $\frac{-4k}{h^2} \in S$ for stability. If we choose $k=\alpha h^2$ then this becomes 
            $-4\alpha \notin S$. Hence the method is not stable for any choice of $\alpha>0$.
            
            \item \textbf{von Neumann Analysis}
            
            Letting $\hat U^n$ be the Fourier transform of the grid function $U^n$ we have 
            $$\hat U^{n+1} = g(\xi)\hat U^n.$$
            If it can be shown that 
            $$|g(\xi)| \leq 1 + \alpha k$$
            with $\alpha$ independent from $\xi$ then it will follow that the method is stable. If we can show the opposite, that
            $$|g(\xi)| > 1 + \alpha k$$
            then the method is unstable. Assuming $$U_j^n = e^{ijh\xi}$$ and $$U_j^{n+1} = g(\xi)e^{ijh\xi},$$ we can express the 
            finite difference scheme in terms of its Fourier transform and solve for $g(\xi)$
            
            \begin{align*}
            &g(\xi)^2e^{ijh\xi} = e^{ijh\xi} + \frac{k}{h^2}g(\xi)(e^{i(j-1)h\xi}-2e^{ijh\xi}+e^{i(j+1)h\xi}) \\
            &\Rightarrow g(\xi)^2 = 1 + g(\xi)\frac{4k}{h^2}(cos(\xi h) - 1) \\
            &\Rightarrow g(\xi)^2 - g(\xi)\frac{4k}{h^2}(cos(\xi h) - 1) - 1 = 0 \\
            &\Rightarrow g(\xi) = \frac{2k}{h^2}(cos(\xi h) - 1) \pm \left(\frac{4k^2}{h^4}(cos(\xi h)^2 + 1\right)^\half.
            \end{align*}
            
            For $\xi= \frac{\pi}{h}, ~ cos(\xi h) = -1$ and so the lesser of the two possible solutions for $g$ becomes
            \[
            g(\xi) = \frac{-4k}{h^2} - \left(\frac{4k^2}{h^4}+1\right)^\half.
            \]
            If we set $k=\alpha h^2$ then this simplifies to
            \[
            g(\xi) = -4\alpha - \left(16\alpha^2 +1\right)^\half < -1
            \]
            for any choice of $\alpha >0$. Hence the method is unstable for any choice of $\alpha >0$.
            \end{enumerate}
        
        
        % Part (c)
        \item Consider the condition for stability specified during the Method of Lines analysis. The eigenvalue farthest from the origin
        was $\lambda_p \approx \frac{-4}{h^2}$, so if we take $k=\alpha h^3$, $\lambda_p k \approx -4\alpha h$ which tends to $0 \in S$
        as the grid is refined, i.e. it approaches $S$ as $h \to 0$. As we saw in class, this allows us to conclude that the method is
        stable.
    
    \end{enumerate}
    
    
\end{homeworkProblem}

\pagebreak

% Problem 2
\begin{homeworkProblem}
   Consider the PDE
    \begin{equation}\label{diffdecay}
    u_t = \kappa u_{xx} - \gamma u,
    \end{equation}
    which models diffusion combined with decay provided $\kappa>0$ and $\gamma>0$.  
    Consider methods of the form
    \begin{equation}\label{ddtheta}
    U_j^{n+1} =U_j^n+{k\kappa\over 2h^2} [U_{j-1}^n-2U_j^n +U_{j+1}^n
    +U_{j-1}^{n+1} - 2U_j^{n+1} + U_{j+1}^{n+1}]
    - k\gamma[(1-\theta)U_j^n + \theta U_j^{n+1}]
    \end{equation}
    where $\theta$ is a parameter.  In particular, if $\theta=1/2$ then the
    decay term is modeled with the same centered-in-time approach as the
    diffusion term and the method can be obtained by applying the Trapezoidal
    method to the MOL formulation of the PDE.   If $\theta=0$ then the decay
    term is handled explicitly.  For more general reaction-diffusion equations
    it may be advantageous to handle the reaction terms explicitly since these
    terms are generally nonlinear, so making them implicit would require solving
    nonlinear systems in each time step (whereas handling the diffusion term
    implicitly only gives a linear system to solve in each time step).
    
    \begin{enumerate}
    \item By computing the local truncation error, show that this method is
    $\bigo(k^p+h^2)$ accurate, where $p=2$ if $\theta = 1/2$ and $p=1$
    otherwise.
    \item Using von Neumann analysis, show that this method is unconditionally
    stable if $\theta \geq 1/2$.
    \item Show that if $\theta = 0$ then the method is stable provided $k\leq
    2/\gamma$, independent of $h$.
    \end{enumerate} 
    
    
    \vskip 1cm
    \textbf{Solution:} \\ 
    
    \begin{enumerate}
    
    % Part (a)
    \item In this derivation of the local truncation error let $u$ denote $u(x,t)$, $u_t$ denote $\pd{t}{u}$, etc.. We Taylor expand
    about $u(x,t)$:
    \begin{align*}
    u(x,t+k) &= u +ku_t+\frac{k^2}{2}u_{tt} +\frac{k^3}{4!}u_{ttt} + O(k^4) \\
    u(x-h,t) &= u -hu_x + \frac{h^2}{2}u_{xx} -\frac{h^3}{6}u_{xxx}+\frac{h^4}{4!}u_{xxxx} +O(h^5) \\
    u(x+h,t) &= u +hu_x + \frac{h^2}{2}u_{xx} +\frac{h^3}{6}u_{xxx}+\frac{h^4}{4!}u_{xxxx} +O(h^5) \\
    u(x+h,t+k) &= u +hu_x +ku_t +\half(h^2u_{xx}+k^2u_{tt} + 2hku_{xt}) + \frac{1}{6}(h^3u_{xxx}+k^3u_{ttt} + 3hk^2u_{xtt} +
    3h^2ku_{xxt}) + \dots \\
    u(x-h,t+k) &= u -hu_x +ku_t +\half(h^2u_{xx}+k^2u_{tt} + 2hku_{xt}) + \frac{1}{6}(-h^3u_{xxx}+k^3u_{ttt} - 3hk^2u_{xtt} +
    3h^2ku_{xxt}) + \dots.
    \end{align*}
    
    Substituting these into the finite difference equation and making obvious cancellations yields
    \begin{align*}
    \tau(x,t) &= (u_t - \kappa u_{xx} + \gamma u) + \frac{k}{2}u_{tt} + \frac{k^2}{6}u_{ttt} - \frac{\kappa h^2}{12}u_{xxxx}
    -\frac{k\kappa}{2}u_{xxt} + \gamma\theta ku_t \\ 
    &+ \frac{\theta \gamma k^2}{2}u_{tt} + O(h^4 + k^3) \\
     &= \frac{k}{2}\pd{t}{u_t -\kappa u_xx + 2\gamma \theta u} + \frac{k^2}{6}u_{ttt} - \frac{\kappa h^2}{12}u_{xxxx} + O(h^4 +k^3).
    \end{align*}
    If $\theta =\half$ we can use the PDE to cancel the $O(k)$ term in the local truncation error, giving overall $O(k^2 + h^2)$
    accuracy. Otherwise this term persists and the method is only $O(k + h^2)$ accurate.
    \pagebreak
    
    % Part (b)
    \item We proceed in a similar manner as before, we substitute $e^{ijh\xi}$ for $U_j^n$, $g(\xi)e^{ijh\xi}$ for $U_j^{n+1}$, etc.
    into the finite difference scheme to determine $g(\xi)$
    \begin{align*}
    g(\xi)e^{ijh\xi} &= e^{ijh\xi}+ e^{ijh\xi}\frac{2k\kappa}{2h^2}\left(cos(\xi h) - 1 + g(\xi)(cos(\xi h) - 1) \right) -
    k\gamma e^{ijh\xi}(1-\theta + \theta g(\xi)) \\
    \Rightarrow g(\xi) &= \frac{h^2(1-k\gamma + k\gamma \theta) +k\kappa (cos(\xi h)-1)}{h^2(1+k\gamma \theta) + 
    k \kappa(1-cos(\xi h))}.
    \end{align*}
    Letting $C = k \kappa(1-cos(\xi h))$, we see that $C \geq 0$ for any $\xi$ and $g$ can be written
    \[
    g(\xi) = 1 - \frac{h^2k\gamma + 2C}{h^2(1+k\gamma \theta) + C}.
    \]
    It is easy to see that this quantity will always be less than $1$, so the only way the method could be unstable is if $g(\xi)<-1$.
    For this to occur we would need to have
    \begin{align*}
    \frac{h^2k\gamma + 2C}{h^2(1+k\gamma \theta) + C} &\geq 2 \\
    \Rightarrow \quad h^2k\gamma + 2C &\geq 2 (h^2(1+k\gamma \theta) + C) \\
    \Rightarrow \quad k\gamma &\geq 2(1+k\gamma \theta) \\
    \Rightarrow \quad k\gamma(1-2\theta) &\geq 2.
    \end{align*}
    If $\theta \geq \half$ then $1-2\theta \leq 0 ~\Rightarrow~ k\gamma(1-2\theta) \leq ~0 ~\leq2 \quad \forall k>0$. Hence the method is
    unconditionally stable for $\theta \geq \half$.
    
    % Part (c)
    \item
    If $\theta =0$ then for $g(\xi)$ as above, $|g(\xi)|\leq 1$ if 
    $k\gamma(1-2\theta) = k\gamma ~ \leq 2 ~ \Rightarrow k~\leq~\frac{2}{\gamma}.$ Note that this restriction holds independent of $h$.
    \end{enumerate}
    
\end{homeworkProblem}

\pagebreak

% Problem 3
\begin{homeworkProblem}
    \begin{enumerate} 
    \item The m-file \verb+heat_CN.m+ from the book repository
    solves the heat equation $u_t = \kappa u_{xx}$  (with $\kappa = 0.02$)
    using the Crank-Nicolson method.
    
    Run this code, and by changing the number of grid points, confirm that it is
    second-order accurate.  (Observe how the error at some fixed time such as $T=1$
    behaves as $k$ and $h$ go to zero with a fixed relation between $k$ and $h$,
    such as $k = 4h$.)
    
    Note that in order for the time step to evenly define the time interval the
    way this code is set up, you need to specify values such as $m=19, 39, 79$.
    (The number of interior grid points.)
    
    Produce a log-log plot of the error versus $h$.
    
    \item Modify this code to produce a new version that
    implements the TR-BDF2 method on the same problem.  Test it to confirm that
    it is also second order accurate.  Explain how you determined the proper
    boundary conditions in each stage of this Runge-Kutta method.
    
    \item Modify the code to produce a new m-file or Python code \verb+heat_FE+ that
    implements the forward Euler explicit 
    method on the same problem.  Test it to confirm that
    it is $\bigo(h^2)$ accurate as $h\goto 0$ provided when $k = 24 h^2$ is
    used, which is within the stability limit for $\kappa = 0.02$.  Note how
    many more time steps are required than with Crank-Nicolson or TR-BDF2,
    especially on finer grids.
    
    \item Test \verb+heat_FE+ with $k = 26 h^2$, for which it should be
    unstable.  Note that the instability does not become apparent until about
    time 4.5 for the parameter values $\kappa = 0.02,~ m=39,~\beta = 150$.
    Explain why the instability takes several hundred time steps to appear, and
    why it appears as a sawtooth oscillation. 
    
    {\bf Hint:} What wave numbers $\xi$ are growing exponentially for these
    parameter values?  What is the initial magnitude of the most unstable
    eigenmode in the given initial data?  The expression (E.30) for the Fourier
    transform of a Gaussian may be useful.
    
    \end{enumerate}

    \vskip 1cm
    \textbf{Solution:} \\
    Note: All trials are run from $t=0$ to final time $t=1$ unless explicitly stated otherwise.
    
    \begin{enumerate}
    
    % Part (a)
    \item To test \verb+heat_CN.m+ I observed how the error behaved at $T=1$ as $k$ and $h$ were refined for $k=4h$.
    
    \begin{figure}[!ht]
    \centering
    \includegraphics[width=.6\textwidth]{586hw4CNerr3a}
    \caption{log-log plot of error using Crank-Nicolson method}
    \label{fig:CN3a}
    \end{figure}
    
    As is demonstrated in Figure \ref{fig:CN3a} the Crank-Nicolson method exhibits second-order accuracy when applied to the heat
    equation with $\kappa=0.02$. The numerically approximated order of accuracy was $2.06$.
    
    
    % Part (b)
    \item Next, to test the TR-BDF2 method I created, I followed the same procedure as in (a) with all parameters the same as for that
    test.
    
    \begin{figure}[!ht]
    \centering
    \includegraphics[width=.6\textwidth]{586hw4trbdf2err3}
    \caption{log-log plot of error using TR-BDF2 method}
    \label{fig:BDF3b}
    \end{figure}
    
    Figure \ref{fig:BDF3b} illustrates that the method is achieving second order accuracy on the heat equation as well. The boundary
    conditions for the first step of the two-stage Runge-Kutta method were the same as in the Crank-Nicolson method, except with
    the boundary conditions evaluated at time $t + \frac{k}{2}$ rather than at $t$ since we are only taking a half-step at the first
    stage. In the second stage the boundary conditions were determined from the following equation
    \[
    U_j^{n+1} = \frac{1}{3}(4U^*_j-U_j^n+\frac{k\kappa}{h^2}(U^{n+1}_{j-1}-2U^{n+1}_{j}+U^{n+1}_{j+1})).
    \]
    Letting $j=1$ and $j=m$ we find that the first and last components (first and mth) of the right-hand-side vector need to have added
    to them the values at the left and right boundaries (multiplied by $\frac{k\kappa}{3h^2}$) at time $t+k$, respectively.The
    numerically approximated order of accuracy was $2.00$. See the m-file \verb+heat_TRBDF2.m+.
    
    % Part (c)
    \item After modifying \verb+heat_CN.m+ to use the Forward Euler method to advance the numerical solutions forward in time I tested
    its accuracy in a similar manner as before, but setting $k=24h^2$.
    
    \begin{figure}[!ht]
    \centering
    \includegraphics[width=.6\textwidth]{586hw4FEerr3}
    \caption{log-log plot of error using Forward Euler method}
    \label{fig:FE3c}
    \end{figure}
    
    Figure \ref{fig:FE3c} demonstrates that the method achieved second order accuracy. The numerically approximated order of accuracy
    was 2.00. See the m-file \verb+heat_FE.m+.
    
    % Part (d)
    \item For this problem I used final time $t=5$ instead of $t=1$. 
    Setting $k=26h^2$ causes the method of lines approach using Forward Euler to discretize the time derivative in the heat
    equation to be unstable.
    
    \begin{figure}[!ht]
    \centering
    \includegraphics[width=.45\textwidth]{586hw4FEunstable1} ~
    \includegraphics[width=.45\textwidth]{586hw4FEunstable2} ~
    \includegraphics[width=.45\textwidth]{586hw4FEunstable3}
    \caption{Evolution of instability in Forward Euler Method of Lines (left: 240 time steps, middle: 260, right: 280}
    \label{fig:FE3d}
    \end{figure}
    
    In Figure \ref{fig:FE3d} we see how the instability progresses as more time steps are taken when 41 grid points are used. Notice
    how the instability does not become noticeable until after roughly 260-280 time steps have been taken. The instability gets
    much worse if time is progressed further (see the attached Matlab notebook).
    
    Recall from the earlier von Neumann analysis that if we consider how the method works on a single wavenumber by setting
    $U_j^n = e^{ijh\xi}$ 
    and $U^{n+1}_j = g(\xi)e^{ijh\xi}$, then the method will be unstable if $|g(\xi)|>1$. 
    Substituting the expressions above into the method and solving for $g(\xi)$ yields
    \[
    g(\xi) = 1 + \frac{2k\kappa}{h^2}(cos(\xi h) - 1).
    \]
    Observe that if we let $k=26h^2$ and $\xi=\frac{\pi}{h}$ (the largest wavenumber representable on our grid), then 
    $g(\xi) = -1.08$. Thus for the wavenumber $\xi=\frac{\pi}{h}$, after $n$ time steps the method will give 
    $\hat U^{n}_j = (-1.08)^n \hat U^0_j$, where $\hat U^{n}_j$ is the Fourier transform of $U^n_j$. Using Parseval's identity, the
    initial magnitude of this unstable eigenmode is simply the magnitude of the Fourier transform of the initial condition, evaluated
    at $\xi=\frac{\pi}{h}$. Following the hint and using the fact that $u(x,0) = u_0(x) = e^{-\beta(x-0.4)^2}$, we have 
    \[
    |\hat u_0(\xi)| = \frac{1}{\sqrt{2\beta}}e^{\frac{-\xi^2}{4\beta}}.
    \]
    For $\beta ~=~ 150,~ \xi ~=~ \frac{\pi}{h} ~=~ 40\pi$, $|\hat u_(\xi)| \approx 2.14 \times 10^{-13}$. Since 
    $(1.08)^n \cdot \approx 2.14 \times 10^{-13}$ does not reach $O(10^{-3}$ until some time after $n=200$ the exponential growth
    in this eigenmode is not noticed until $t$ becomes fairly large. The reason the instability manifests itself as a sawtooth
    oscillation is because a sawtooth oscillation is exactly the way the eigenmode corresponding to $\xi=\frac{\pi}{h}$ is represented
    on our chosen grid. It oscillates at a rate exactly matching our grid spacing and so the grid points only capture its points of
    highest and lowest amplitude, one after another.
    
    \end{enumerate}
    

\end{homeworkProblem}

\begin{homeworkProblem}
    \begin{enumerate}

    \item Modify \verb+heat_CN.m+  (or \verb+heat_CN.py+)
    to solve the heat equation for
    $-1\leq x \leq 1$ with step function  initial data
    \begin{equation} \label{9.3a}
    u(x,0) = \begin{choices}  1 \when x<0\\  0 \when x\geq 0. \end{choices}
    \end{equation} 
    With appropriate Dirichlet boundary conditions, the exact solution is
    \begin{equation} \label{9.3b}
    u(x,t) = \half \, \text{erfc} \left(x / \sqrt{4 \kappa t}\right)
    \end{equation} 
    for $t>0$, where erfc is the {\em complementary error function}
    \[
    \text{erfc}(x) = \frac{2}{\sqrt{\pi}} \int_x^\infty e^{-z^2}\,dz.
    \]
    Note that {\tt erfc} is a built-in Matlab function and in Python you could
    use \verb+scipy.special.erfc+.
    
    \begin{enumerate}
    \item
    Use $\kappa = 0.02$ and
    test this method using $m=39$ and $k = 10h$ for $-1\leq x \leq 1$.  
    Note that there is an initial rapid transient decay of the high
    wave numbers that is not captured well with this size time step.
    
    \item
    How small do you need to take the time step to get reasonable results?
    For a suitably small time step, explain why you get much better results by
    using $m=38$ than $m=39$.  What is the observed order of accuracy as $k\goto
    0$ when $k = \alpha h$ with $\alpha$ suitably small and $m$ even?
    
    Hint: You might also see what you get if you redefine the initial data so
    that $u(0,0)=1/2$ with $u(x,0)$ the same everywhere else, rather than
    $u(0,0) = 0$.
    
    \end{enumerate}
    
    \item Modify your TR-BDF2 solver from Problem 3
    to solve the heat equation for
    $-1\leq x \leq 1$ with step function  initial data as above.
    Test this routine using $k=10h$ and estimate the order of accuracy as
    $k\goto 0$ with $m$ even.  Why does the TR-BDF2 method work better than
    Crank-Nicolson?
    
    \end{enumerate} 

    \vskip 1cm
    \textbf{Solution:} \\
    
    \begin{enumerate}
    
    % Part (a)
    \item The modified version of \verb+heat_CN.m+ is stored in \verb+heat_CN_alt_BCs.m+.
        \begin{enumerate}
        
        % Part (i)
        \item First we test the method using the specified parameters.
        \begin{figure}[!ht]
        \centering
        \includegraphics[width=.6\textwidth]{586hw4CN_plot4a}
        \caption{Crank-Nicolson Method with $k=10h$}
        \label{fig:CN4ai}
        \end{figure}
        
        Figure \ref{fig:CN4ai} shows the numerical solution generated by the Crank-Nicolson method when $k=10h$. Notice the oscillation
        which occurs at $x=0$. The maximum error at $t=1$ is $0.204$.
        
        % Part (ii)
        \item After multiple numerical tests using even and odd numbers of grid points, $\alpha = 2$ seems to be an appropriate
        restriction to get reasonable results. That is for $\alpha=2$ the error in the approximation at $t=1$ behaves predictably
        as $k$ is refined. If $\alpha$ is taken to be much larger than $2$, $2.5$, say, then as the grid is refined, eventually the error
        stops decreasing and starts leveling off instead. 
        
        \begin{figure}[!ht]
        \centering
        \includegraphics[width=.45\textwidth]{586hw4CN_plot4aalpha6} ~
        \includegraphics[width=.45\textwidth]{586hw4CN_plot4aalpha5}
        \caption{Crank-Nicolson method with left:$k=6h$, right:$k=5h$}
        \label{fig:CN4aii}
        \end{figure}
        
        Figure \ref{fig:CN4aii} shows an example of how the solution seems able to resolve the transient for $\alpha=5$, but not
        $\alpha=6$. However, if the grid is further refined, we find that $\alpha=5$ is too lax a condition as mentioned above, and the
        error starts to worsen.
        
        \begin{figure}[!ht]
        \centering
        \includegraphics[width=.45\textwidth]{586hw4CN_errploteven} ~
        \includegraphics[width=.45\textwidth]{586hw4CN_errplotodd}
        \caption{Crank-Nicolson method with (left:even, right:odd) number of grid points}
        \label{fig:CN4aiieo}
        \end{figure}
        
        In Figure \ref{fig:CN4aiieo} we have the maximum error at $t=1$ using even and odd numbers of grid points. The computed
        orders of accuracy were $2.03$ using an even number and $1.01$ using an odd number. The reason for the disparity in accuracy
        between the two cases is that when an odd number of grid points are used, one of the points lies directly at $0$. This causes
        the initial condition to be less accurately represented on the odd grid than the even one. If we follow the hint and set
        $u(0,0) = \half$ then the method attains second order accuracy on both the even and odd grids. This tells us that the difference
        in performance must be due to the way the initial condition is resolved.
        
        \end{enumerate}
    
    
    % Part (b)
    \item The TR-BDF2 solver allows us to take a larger time step than we could using the Crank-Nicolson method, resulting in faster
    overall performance. As noted in part (a)(ii) of the problem, it is essential that we choose an even number of grid points to
    better resolve the initial condition and get second order convergence.
    
    \begin{figure}[!ht]
    \centering
    \includegraphics[width=.6\textwidth]{586hw4TRBDF2_errploteven}
    \caption{TR-BDF2 Method error with $k=10h$}
    \label{fig:TRBDF4b}
    \end{figure}
    
    In Figure \ref{fig:TRBDF4b} we show the behavior of the maximum error in the TR-BDF2 method as k is decreased.
    This method works better than Crank-Nicolson because it is both L-stable and A-stable, whereas the implicit method relied upon
    in the Crank-Nicolson method, the Trapezoid rule, is merely A-stable. Note that the TR-BDF2 method also uses the Trapezoid
    method in one of its stages, but it is L-stable nevertheless. The rapid initial transient in the solution leads
    to a loss in stability in the Crank-Nicolson method, but not TR-BDF2, since rapid initial
    transients are known to be troublesome for the trapezoidal method by itself.
    The numerically approximated order of accuracy is $2.59$. The code for the modified method can be found in \verb+TRBDF2_alt.m+.
    
    \end{enumerate}
    
    
    
    \end{homeworkProblem}

\pagebreak
~~
\vskip 10cm
\centering
\huge \textbf{Appendix: Code}


\end{document}

