\documentclass{article}

\usepackage{fancyhdr}
\usepackage{extramarks}
\usepackage{amsmath}
\usepackage{amsthm}
\usepackage{amsfonts}
\usepackage{amssymb}
\usepackage{graphicx}
\usepackage{caption,subcaption}
\usepackage{subfig}

%
% Basic Document Settings
%

\topmargin=-0.45in
\evensidemargin=0in
\oddsidemargin=0in
\textwidth=6.5in
\textheight=9.0in
\headsep=0.25in

\linespread{1.1}

\pagestyle{fancy}
\lhead{\hmwkAuthorName}
\chead{\hmwkClass:\ \hmwkTitle}
\rhead{\firstxmark}
\lfoot{\lastxmark}
\cfoot{\thepage}

\renewcommand\headrulewidth{0.4pt}
\renewcommand\footrulewidth{0.4pt}

\setlength\parindent{0pt}

%
% Create Problem Sections
%

\newcommand{\enterProblemHeader}[1]{
    \nobreak\extramarks{}{Problem \arabic{#1} continued on next page\ldots}\nobreak{}
    \nobreak\extramarks{Problem \arabic{#1} (continued)}{Problem \arabic{#1} continued on next page\ldots}\nobreak{}
}

\newcommand{\exitProblemHeader}[1]{
    \nobreak\extramarks{Problem \arabic{#1} (continued)}{Problem \arabic{#1} continued on next page\ldots}\nobreak{}
    \stepcounter{#1}
    \nobreak\extramarks{Problem \arabic{#1}}{}\nobreak{}
}

\setcounter{secnumdepth}{0}
\newcounter{partCounter}
\newcounter{homeworkProblemCounter}
\setcounter{homeworkProblemCounter}{1}
\nobreak\extramarks{Problem \arabic{homeworkProblemCounter}}{}\nobreak{}

%
% Homework Problem Environment
%
% This environment takes an optional argument. When given, it will adjust the
% problem counter. This is useful for when the problems given for your
% assignment aren't sequential. See the last 3 problems of this template for an
% example.
%
\newenvironment{homeworkProblem}[1][-1]{
    \ifnum#1>0
        \setcounter{homeworkProblemCounter}{#1}
    \fi
    \section{Problem \arabic{homeworkProblemCounter}}
    \setcounter{partCounter}{1}
    \enterProblemHeader{homeworkProblemCounter}
}{
    \exitProblemHeader{homeworkProblemCounter}
}

%
% Homework Details
%   - Title
%   - Due date
%   - Class
%   - Section/Time
%   - Instructor
%   - Author
%

\newcommand{\hmwkTitle}{Homework 5}
\newcommand{\hmwkDueDate}{May 26, 2015}
\newcommand{\hmwkClass}{Amath 586}
\newcommand{\hmwkAuthorName}{Brian de Silva}

%
% Title Page
%

\title{
    \vspace{2in}
    \textmd{\textbf{\hmwkClass:\ \hmwkTitle}}\\
    \normalsize\vspace{0.1in}\small{Due\ on\ \hmwkDueDate\ }\\
    \vspace{3in}
}

\author{\textbf{\hmwkAuthorName}}
\date{}

\renewcommand{\part}[1]{\textbf{\large Part \Alph{partCounter}}\stepcounter{partCounter}\\}

%
% Various Helper Commands
%

% Useful for algorithms
\newcommand{\alg}[1]{\textsc{\bfseries \footnotesize #1}}

% For derivatives
\newcommand{\deriv}[1]{\frac{\mathrm{d}}{\mathrm{d}x} (#1)}

% For partial derivatives
\newcommand{\pd}[2]{\frac{\partial}{\partial #1} (#2)}

\newcommand{\pdd}[2]{\frac{\partial #1}{\partial #2}}

% Integral dx
\newcommand{\dx}{\mathrm{d}x}

% Alias for the Solution section header
\newcommand{\solution}{\textbf{\large Solution}}

% Probability commands: Expectation, Variance, Covariance, Bias
\newcommand{\E}{\mathrm{E}}
\newcommand{\Var}{\mathrm{Var}}
\newcommand{\Cov}{\mathrm{Cov}}
\newcommand{\Bias}{\mathrm{Bias}}

% Some useful macros
\input{./macros.tex}

\begin{document}

\maketitle

\pagebreak

% Problem 1
\begin{homeworkProblem}
    Let $U = [U_0,~U_1,~\ldots,~U_m]^T$ be a vector of function values at
equally spaced points on the interval $0\leq x \leq 1$, and suppose the
underlying function is periodic and smooth.  Then we can approximate 
the first derivative $u_x$ at all of these points by $DU$, where $D$ is
circulant matrix such as
\[
D_- = \frac 1 h \brm 1&&&&-1\\ -1&1\\ &-1&1\\ &&-1&1\\ &&&-1&1\erm,  \qquad
D_+ = \frac 1 h \brm -1&1\\ &-1&1\\ &&-1&1\\ &&&-1&1\\ 1&&&&-1\erm
\]
for first-order accurate one-sided approximations or
\[
D_0 = \frac 1 {2h} \brm 0&1&&&-1\\ -1&0&1\\ &-1&0&1\\ &&-1&0&1\\ 1&&&-1&0\erm
\]
for a second-order accurate centered approximation.  (These are illustrated
for a grid with $m+1=5$ unknowns and $h=1/5$.)


The advection equation $u_t + au_x=0$ on the interval $0\leq x \leq
1$ with periodic boundary conditions  
gives rise to the MOL discretization $U'(t) = -aDU(t)$
where $D$ is one of the matrices above.

\begin{enumerate} 
\item Discretizing $U' = -aD_-U$ by forward Euler gives the first order
upwind method
\[
U_j^{n+1} = U_j^n - \frac{ak}{h} (U_j^n - U_{j-1}^n),
\]
where the index $i$ runs from 0 to $m$ with addition of indices performed
mod $m+1$ to incorporate the periodic boundary conditions.

Suppose instead we discretize the MOL equation by the second-order Taylor
series method, 
\[
U^{n+1} = U^n - akD_-U^n + \half (ak)^2 D_-^2 U^n.
\]
Compute $D_-^2$ and also write out the formula for $U_j^n$ that results from
this method.  

\item How accurate is the method derived in part (a) compared to the
Beam-Warming method, which is also a 3-point one-sided method?

\item Suppose we make the method more symmetric:
\[
U^{n+1} = U^n - \frac{ak}{2} (D_+ +D_-)U^n + \half (ak)^2 D_+D_- U^n.
\]
Write out the formula for $U_j^n$ that results from this method.  
What standard method is this?

\end{enumerate}
    
    
    \vskip 1cm
    \textbf{Solution:} \\
    
    \begin{enumerate}
    
    % Part (a)
    \item $D^2_-$ is easily computed:
    \begin{equation*}
    D^2_- = \frac{1}{h^2} \brm 1&& & & & 1&-2\\
                                -2&1& & & & & 1 \\
                                1&-2&1&\\
                                &1&-2&1&\\
                                &&\ddots&\ddots&\ddots&\\
                                &&&1&-2&1&\\
                                &&&&1&-2&1
                                \erm .
    \end{equation*}
    The formula that results from this method is given by
    \begin{equation*}
    U_j^{n+1} =U_j^n - \frac{ak}{h}\left(U^n_j-U^n_{j-1}\right) + \half \left(\frac{ak}{h}\right)^2\left( U_j^n-2U_{j-1}^n+U_{j-2}\right),
    \quad j=1,2,\dots
    \end{equation*}
    where we identify $U_0^n$ with $U_{m+1}^n$ and $U_{-1}^n$ with $U_{m}^n$  because of the periodic boundary conditions.
    
    % Part (b)
    \item This method is only $O(k^2 + h)$ as it uses a second order discretization in time and a first order one in space, whereas
    the Beam-Warming method is $O(k^2+h^2)$ accurate. This method is the worse of the two 3-point one-sided methods.
    
    % Part (c)
    \item The formula that results from making this method symmetric is
    \begin{equation*}
    U^{n+1}_j = U^n_j - \frac{ak}{2h}\left(U^n_{j+1} - U^n_{j-1} \right) +
    \half \left(\frac{ak}{h}\right)^2\left(U^n_{j-1}-2U^n_j+U^n_{j+1}\right),
    \end{equation*}
    which is just the Lax-Wendroff method.
    
    
    \end{enumerate}

\end{homeworkProblem}


% Problem 2
\begin{homeworkProblem}
    \begin{enumerate} 
\item
Produce a plot similar to those shown in Figure 10.1 for the upwind method
(10.21) with the same values of $a=1$, $h=1/50$ and $k=0.8h$
used in that figure.

\item Produce the corresponding plot if the one-sided method (10.22) is
instead used with the same values of $a,~h$, and $k$.
\end{enumerate}
    
    
    \vskip 1cm
    \textbf{Solution:} \\
    
    \begin{figure}
    \centering
    \includegraphics[width=.65\textwidth]{2a}
    \caption{Eigenvalues of $A_{\epsilon}$ with $\epsilon = \frac{ah}{2}$} 
    \label{fig:2a}
    \end{figure}
    
    \begin{figure}
    \centering
    \includegraphics[width=.65\textwidth]{2b}
    \caption{Eigenvalues of $A_{\epsilon}$ with $\epsilon = -\frac{ah}{2}$}
    \label{fig:2b}
    \end{figure}
    
    \begin{enumerate}
    
    % Part (a)
    \item Using the upwind method and the same values of $a=1$, $h=1/50$ and $k=0.8h$ as in Figure 10.1 we produce the plot 
    shown in Figure \ref{fig:2a}.
    
    
    % Part (b)
    \item Method (10.22) and the same values as above produces the plot shown in Figure \ref{fig:2b}.
    
    \end{enumerate}
    
    See the appendix for the Matlab code that generated these plots.
    
\end{homeworkProblem}

\pagebreak

% Problem 3
\begin{homeworkProblem}
    Suppose $a>0$ and consider the following {\it skewed leapfrog} method for
solving the advection equation $u_t + au_x = 0$:
\[
U_j^{n+1} = U_{j-2}^{n-1}  - \left(\frac{ak}{h} - 1\right) (U_j^n -
U_{j-2}^n).
\]
The stencil of this method is
\vskip 5pt
\hfil\includegraphics[width=1.5in]{skewedlfstencil.png}\hfil
\vskip 5pt

Note that if $ak/h \approx 1$ then this stencil roughly follows the
characteristic of the advection equation and might be expected to be more
accurate than standard leapfrog.  (If $ak/h = 1$ the method is exact.)


\begin{enumerate}

\item What is the order of accuracy of this method?

\item For what range of Courant number $ak/h$ does this method satisfy the
CFL condition?

\item Show that the method is in fact stable for this range of Courant
numbers by doing von Neumann analysis.  
{\bf Hint:} Let $\gamma(\xi) = e^{i\xi h}g(\xi)$ and show that $\gamma$
satisfies a quadratic equation closely related to the equation (10.34)
that arises from a von Neumann analysis of the leapfrog method.

\end{enumerate}
    
    \vskip 1cm
    \textbf{Solution:} \\
    
    \begin{enumerate}
    
    % Part (a)
    \item We need the following Taylor expansions to compute the truncation error for this method:
    \begin{align*}
    u(x,t+k) &= u + ku_t+\frac{k^2}{2}u_{tt} + \frac{k^3}{3!}u_{ttt} + O(k^4), \\
    u(x-2h,t) &= u-2hu_x+2h^2u_{xx}-\frac{4}{3}h^3u_{xxx}+O(h^4), \\
    u(x-2h,t-k) &= u-2hu_x-ku_t+\half(4h^2u_{xx}+k^2u_{tt}+4khu_{xt})\\
    &-\frac{1}{6}(8h^3u_{xxx}+k^3u_{ttt}+6hk^2u_{xtt}+12h^2ku_{xxt})+\dots .
    \end{align*}
    
    Rewriting the method, we obtain an expression for the truncation error:
    \begin{equation*}
    \tau^n=\frac{U^{n+1}_j - U^{n-1}_{j-2}}{k} + \frac{1}{k}\left(\frac{ak}{h}-1\right)(U^n_j-U^n_{j-2}).
    \end{equation*}
    
    Substituting in the above Taylor expansions, cancelling terms, and using the PDE to simplify the resulting equation further, we get
    \begin{equation*}
    \tau^n = \frac{k^2}{3}u_{ttt}+hku_{ttx}+2h^2u{xxt}+\frac{4}{3}ah^2u_{xxx} +\dots = O((h+k)^2).
    \end{equation*}
    
    Hence the method is second order accurate in both space and time. (Note: although I did above the calculations by hand, 
    I checked my answer using a Mathematica notebook which can be found in the Appendix).
    
    % Part (b)
    \item Using the skewed leapfrog formula, we can trace the dependence of an approximation $U^n_j$ back to the points
    $U^0_{j-2n}, U^0_{j-2(n-1)}, \dots, U^0_j$. Fixing $\frac{k}{h}=r$ we find that as we refine the grid, the numerical domain of
    dependence for a point $(X,T)$ corresponding to $U^n_j$ converges to $[X-2T/r,X]$. For $a>0$ the domain of dependence of the PDE
    at $(X,T)$ is $\{X-aT\}$. Hence in order for the method to satisfy the CFL condition, we need that
    $X-2T/r \leq X-aT \leq X$. For $a>0$ this means that we must have $ar\leq2 ~\Rightarrow ~ 0\leq \frac{ak}{h}\leq 2$.
    
    % Part (c)
    \item Let $\nu=\frac{ak}{h}$, the Courant number. Setting $U^n_j = g(\xi)^ne^{ijh\xi}$ in the skewed leapfrog method yields
    \begin{equation*}
    g(\xi)^{n+1}e^{ijh\xi} =g(\xi)^{n-1}e^{i(j-2)h\xi} - g(\xi)^{n}\left(\nu-1\right)\left(e^{ijh\xi}-e^{i(j-2)h\xi} \right).
    \end{equation*}
    Dividing by $g(\xi)^{n-1}e^{i(j-2)h\xi}$ gives
    \begin{equation*}
    g(\xi)^2e^{2ih\xi} = 1 - g(\xi)\left(\nu-1\right)\left(e^{2ih\xi}-1\right).
    \end{equation*}
    Finally, setting $\gamma(\xi)=g(\xi)e^{ijh\xi}$ in the above, we have
    \begin{align*}
    \gamma(\xi)^2 &= 1 - \gamma(\xi)\left(\nu-1\right)\left(e^{ih\xi}-e^{-ih\xi}\right) \\
    \Rightarrow \gamma(\xi)^2 &= 1-2i(\nu-1)sin(h\xi)\gamma(\xi).
    \end{align*}
    Applying the same analysis as in Example 10.4, we see that in order for the method to be stable 
    (i.e. $|\gamma(\xi)|=|g(\xi)e^{ijh\xi}| = |g(\xi)| \leq 1$), we require $|\nu-1|\leq 1 ~\Rightarrow~ 0\leq \nu \leq 2$, the same
    restriction that the CFL condition imposes. Therefore the method is stable for the range of Courant numbers satisfying the CFL
    condition.
    
    
    \end{enumerate}
    
\end{homeworkProblem}


% Problem 4
\begin{homeworkProblem}
    Derive the modified equation (10.45) for the Lax-Wendroff method.
    
    \vskip 1cm
    \textbf{Solution:} \\
    
    The Lax-Wendroff method is given by
    \begin{equation*}
    U^{n+1}_j = U^n_j - \frac{ak}{2h}(U^n_{j+1}-U^n_{j-1}) + \frac{a^2k^2}{2h^2}(U^n_{j-1}-2U^n_j+U^n_{j+1}).
    \end{equation*}
    Supposing that $v(x,t)$ satisfies this difference equation exactly, we substitute it into the equation, Taylor expand about $(x,t)$, and
    simplify to get
    \begin{equation*}
    \left(kv_t+\frac{k^2}{2}v_{tt}+\frac{k^3}{6}v_{ttt} + \frac{k^4}{4!}v_{4t} + O(k^5)\right) + 
    ak\left(v_x+\frac{h^2}{3!}v_{xxx}+O(h^4)\right) -\frac{a^2k^2}{2}\left(v_{xx}+\frac{h^2}{12}v_{4x}+O(h^4)\right)=0.
    \end{equation*}
    
    This can be rewritten as 
    \begin{equation}\label{eq:4a}
    v_t+av_x = \frac{k}{2}\left(a^2v_{xx}-v_{tt}\right) - \frac{1}{6}\left(k^2v_{ttt}+ah^2v_{xxx}\right)+O(k^3+h^3+h^2k).
    \end{equation}
    
    Dropping higher order terms and differentiating with respect to $x$ and $t$ gives
    \begin{align*}
    v_{tx}&= -av_{xx} + \frac{k}{2}\left(a^2v_{xxx}-v_{ttx}\right) - \frac{1}{6}\left(k^2v_{tttx}+ah^2v_{4x}\right), \\
    v_{tt}&= -av_{xt} + \frac{k}{2}\left(a^2v_{xxt}-v_{ttt}\right) - \frac{1}{6}\left(k^2v_{4t}+ah^2v_{xxxt}\right).
    \end{align*}
    
    Combining these, we obtain
    \begin{equation*}
    v_{tt} = a^2v_{xx} - \frac{ak}{2}\left(a^2v_{xxx}-v_{ttx}\right) + \frac{k}{2}\left(a^2v_{xxt}-v_{ttt}\right) + O(k^2 + h^2).
    \end{equation*}
    
    Substituting this into (\ref{eq:4a}) and ignoring higher order terms yields
    \begin{equation*}
    v_t+av_x = \frac{k^2}{4}\left(a^2v_{xxt}-v_{ttt}-a^3v_{xxx}+av_{ttx} \right) - \frac{1}{6}\left(k^2v_{ttt}+ah^2v_{xxx}\right).
    \end{equation*}
    
    Differentiating earlier equations with respect to $x$ and $t$ once more gives
    \begin{align*}
    v_{txx} &= -av_{xxx} + O(k), \\
    v_{ttx} &= a^2v_{xxx}+O(k), \\
    v_{ttt} &= a^2v_{xxt} + O(k) = -a^3(v_{xxx})+O(k).
    \end{align*}
    
    Finally, substituting these into the above and ignoring higher order terms yields
    
    \begin{align*}
    v_t + av_x&=\frac{k^2}{4}\left(a^2v_{xxt} - (a^2v_{xxt}) - a^3v_{xxx} + a(a^2v_{xxx}) \right)
    -\frac{1}{6}\left( k^2(-a^3v_{xxx})+ah^2v_{xxx} \right) \\
    &= -\frac{ah^2}{6}\left( 1-\left(\frac{ak}{h}\right)^2 \right)v_{xxx}.
    \end{align*}
    
    Hence we obtain the desired expression
    \begin{equation*}
    v_t + av_x +\frac{ah^2}{6}\left( 1-\left(\frac{ak}{h}\right)^2 \right)v_{xxx} =0.
    \end{equation*}
    
    
\end{homeworkProblem}



% Problem 5
\begin{homeworkProblem}
    The m-file \verb+advection_LW_pbc.m+ from the book repository
implements the Lax-Wendroff method for
the advection equation on $0\leq x\leq 1$ with 
periodic boundary conditions.

The class repository contains a Python translation, \verb+$AM586/codes/advection_LW_pbc.py+.

You might want to first experiment with these and see how Lax-Wendroff behaves for various grid resolutions.  Note that the way this problem is set up, the solution advects twice around the domain over time 1 and should end up agreeing with the initial data. 

\begin{enumerate}
\item Modify one of these files to 
implement the leapfrog method and verify that this is second order accurate.
Note that you will have to specify two levels of initial data.  For the
convergence test set $U^1_j = u(x_j,k)$, the true solution at time $k$.

\item Modify your code so that the initial data consists of
a wave packet
\[
\eta(x) = \exp(-\beta(x-0.5)^2) \sin(\xi x)
\]
Work out the true solution $u(x,t)$ for this data.
Using $\beta = 100$, $\xi=80$ and $U^1_j = u(x_j,k)$, test that your code
still exhibits second order accuracy for $k$ and $h$ sufficiently small.

\item Using $\beta = 100$, $\xi=150$ and $U^1_j = u(x_j,k)$, estimate the
group velocity of the wave packet computed with leapfrog using $m=199$ and
$k = 0.4h$.  How well does this compare with the  value (10.52) predicted by
the modified equation?  

{\bf Note:} Early editions of the text book had a typo in (10.52).  There should be a factor $\nu^2$ in the denominator.

\end{enumerate}
    
    
    \vskip 1cm
    \textbf{Solution:} \\
    
    \begin{figure}
    \centering
    \includegraphics[width=.65\textwidth]{5aerr}
    \caption{Error at $t=1$ using Leapfrog method}
    \label{fig:5a}
    \end{figure}
    
    \begin{figure}
    \centering
    \includegraphics[width=.65\textwidth]{5berr}
    \caption{Error at $t=1$ using Leapfrog method with alternate initial conditions}
    \label{fig:5b}
    \end{figure}
    
    \begin{enumerate}
    
    % Part (a)
    \item Figure (\ref{fig:5a}) shows the error at $t=1$ of the Leapfrog method as the grid is refined. The numerically approximated order of accuracy was
    $1.99$ so the method is exhibiting second order accuracy.
    
    % Part (b)
    \item For the simple advection equation the solution is simply $\eta(x-at)$ modified so that it is periodic with period 1. As shown in Figure 
    (\ref{fig:5b}), the method is still second order accurate on this new initial data (approximate order of accuracy: 1.96). Notice however that 
    the grid spacing had to be decreased in order
    for the desired second order accuracy to become apparent. For larger values of $h$ and $k$ the method was not converging nearly as quickly.
    
    % Part (c)
    \item Following the peak of the wave packet of the numerical solution, we can approximate its wave speed. We see that it travels a total distance of
    1.752 over 1 unit of time, so the group velocity is roughly 1.75.
    Equation (10.52) uses the modified equation to approximate the group velocity of the wave packet computed with leapfrog as 
    \begin{equation*}
    \pm \frac{acos(\xi h)}{\sqrt(1-\nu^2sin^2(\xi h))} \approx 1.746
    \end{equation*}
    using the given parameters $a=2, \nu=\frac{4}{5}, h=\frac{1}{200}, \xi =150.$ This agrees almost perfectly with the observed group velocity.
    
    \end{enumerate}
    
\end{homeworkProblem}
        

\pagebreak

\hfill

\pagebreak
~~
\vskip 10cm
\centering
\huge \textbf{Appendix: Code}


\end{document}

