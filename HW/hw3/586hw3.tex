\documentclass{article}

\usepackage{fancyhdr}
\usepackage{extramarks}
\usepackage{amsmath}
\usepackage{amsthm}
\usepackage{amsfonts}
\usepackage{amssymb}
\usepackage{graphicx}
\usepackage{caption,subcaption}
\usepackage{subfig}

%
% Basic Document Settings
%

\topmargin=-0.45in
\evensidemargin=0in
\oddsidemargin=0in
\textwidth=6.5in
\textheight=9.0in
\headsep=0.25in

\linespread{1.1}

\pagestyle{fancy}
\lhead{\hmwkAuthorName}
\chead{\hmwkClass:\ \hmwkTitle}
\rhead{\firstxmark}
\lfoot{\lastxmark}
\cfoot{\thepage}

\renewcommand\headrulewidth{0.4pt}
\renewcommand\footrulewidth{0.4pt}

\setlength\parindent{0pt}

%
% Create Problem Sections
%

\newcommand{\enterProblemHeader}[1]{
    \nobreak\extramarks{}{Problem \arabic{#1} continued on next page\ldots}\nobreak{}
    \nobreak\extramarks{Problem \arabic{#1} (continued)}{Problem \arabic{#1} continued on next page\ldots}\nobreak{}
}

\newcommand{\exitProblemHeader}[1]{
    \nobreak\extramarks{Problem \arabic{#1} (continued)}{Problem \arabic{#1} continued on next page\ldots}\nobreak{}
    \stepcounter{#1}
    \nobreak\extramarks{Problem \arabic{#1}}{}\nobreak{}
}

\setcounter{secnumdepth}{0}
\newcounter{partCounter}
\newcounter{homeworkProblemCounter}
\setcounter{homeworkProblemCounter}{1}
\nobreak\extramarks{Problem \arabic{homeworkProblemCounter}}{}\nobreak{}

%
% Homework Problem Environment
%
% This environment takes an optional argument. When given, it will adjust the
% problem counter. This is useful for when the problems given for your
% assignment aren't sequential. See the last 3 problems of this template for an
% example.
%
\newenvironment{homeworkProblem}[1][-1]{
    \ifnum#1>0
        \setcounter{homeworkProblemCounter}{#1}
    \fi
    \section{Problem \arabic{homeworkProblemCounter}}
    \setcounter{partCounter}{1}
    \enterProblemHeader{homeworkProblemCounter}
}{
    \exitProblemHeader{homeworkProblemCounter}
}

%
% Homework Details
%   - Title
%   - Due date
%   - Class
%   - Section/Time
%   - Instructor
%   - Author
%

\newcommand{\hmwkTitle}{Homework 3}
\newcommand{\hmwkDueDate}{April 30, 2015}
\newcommand{\hmwkClass}{Amath 586}
\newcommand{\hmwkAuthorName}{Brian de Silva}

%
% Title Page
%

\title{
    \vspace{2in}
    \textmd{\textbf{\hmwkClass:\ \hmwkTitle}}\\
    \normalsize\vspace{0.1in}\small{Due\ on\ \hmwkDueDate\ }\\
    \vspace{3in}
}

\author{\textbf{\hmwkAuthorName}}
\date{}

\renewcommand{\part}[1]{\textbf{\large Part \Alph{partCounter}}\stepcounter{partCounter}\\}

%
% Various Helper Commands
%

% Useful for algorithms
\newcommand{\alg}[1]{\textsc{\bfseries \footnotesize #1}}

% For derivatives
\newcommand{\deriv}[1]{\frac{\mathrm{d}}{\mathrm{d}x} (#1)}

% For partial derivatives
\newcommand{\pd}[2]{\frac{\partial}{\partial #1} (#2)}

\newcommand{\pdd}[2]{\frac{\partial #1}{\partial #2}}

% Integral dx
\newcommand{\dx}{\mathrm{d}x}

% Alias for the Solution section header
\newcommand{\solution}{\textbf{\large Solution}}

% Probability commands: Expectation, Variance, Covariance, Bias
\newcommand{\E}{\mathrm{E}}
\newcommand{\Var}{\mathrm{Var}}
\newcommand{\Cov}{\mathrm{Cov}}
\newcommand{\Bias}{\mathrm{Bias}}

% Some useful macros
\input{./macros.tex}

\begin{document}

\maketitle

\pagebreak

% Problem 1
\begin{homeworkProblem}
    Let $g(x)=0$ represent a system of $s$ nonlinear equations in $s$ unknowns,
    so $x\in\reals^s$ and $g: \reals^s \goto \reals^s$.  A vector $\bar
    x\in\reals^s$ is a {\em fixed point} of $g(x)$ if 
    \begin{equation}\label{a}
    \bar x = g(\bar x).
    \end{equation}
    One way to attempt to compute $\bar x$ is with {\em fixed point iteration}:
    from some starting guess $x^0$, compute
    \begin{equation}\label{b}
    x^{j+1} = g(x^j)
    \end{equation}
    for $j=0,~1,~\ldots$.
    
    \begin{enumerate}
    \item Show that if there exists a norm $\|\cdot\|$ such that $g(x)$ is
    Lipschitz continuous with constant $L<1$ in a neighborhood of $\bar x$, then
    fixed point iteration converges from any starting value in this
    neighborhood.
    {\bf Hint:} Subtract equation \eqn{a} from \eqn{b}.
    
    \item Suppose $g(x)$ is differentiable and let $g'(x)$ be the $s\times s$
    Jacobian matrix.  Show that if the condition of part (a) holds then
    $\rho(g'(\bar x)) < 1$, where $\rho(A)$ denotes the spectral radius of a
    matrix.
    
    \item Consider a predictor-corrector method (see Section 5.9.4) consisting
    of forward Euler as the predictor and backward Euler as the corrector, and
    suppose we make $N$ correction iterations, i.e., we set
    \begin{tabbing}
    xxxxxxxxx\=xxxx\=\kill\\
    \>$\hat U^0 = U^n + kf(U^n)$\\
    \>for $j = 0,~1,~\ldots,~N-1$\\
    \>\>$\hat U^{j+1} = U^n + kf(\hat U^j)$\\
    \>\>end\\
    \>$U^{n+1} = \hat U^N$.
    \end{tabbing}
    Note that this can be interpreted as a fixed point iteration for solving the
    nonlinear equation
    \[
    \unp = U^n + kf(\unp)
    \]
    of the backward Euler method.  Since the backward Euler method is implicit
    and has a stability region that includes the entire left half plane, as
    shown in Figure 7.1(b), one might hope that this predictor-corrector method
    also has a large stability region.
    
    Plot the stability region $S_N$
    of this method for $N=2,~5,~10,~20$ (perhaps using
    {\tt plotS.m} from the webpage) and observe that in fact the stability
    region does not grow much in size.
    
    \item Using the result of part (b), show that the fixed point iteration
    being used in the predictor-corrector method of part (c) can only be
    expected to converge if $|k\lambda| < 1$ for all eigenvalues $\lambda$ of the
    Jacobian matrix $f'(u)$.  
    
    \item Based on the result of part (d) and the shape of the stability region
    of Backward Euler, what do you expect the stability region $S_N$ of part (c)
    to converge to as $N\goto\infty$?
    
    \end{enumerate}
    
    
    \vskip 1cm
    \textbf{Solution:} \\
    \begin{enumerate}
        
        % Part (a)
        \item Suppose there exists a norm $\|\cdot\|$ such that $g(x)$ is
    Lipschitz continuous with constant $L<1$ in a neighborhood, $N$, of $\bar x$. Let $x^0 \in N$ and let $x^1 = g(x^0), ~x^2 = g(x^1)
    = g(g(x^0))$, etc.. Notice that $g(N) \subset N$ since for any $x \in N, |g(x) - g(\bar x)| \leq L|x-\bar x| \leq |x - \bar x| \Longrightarrow 
    g(x) \in N$. Hence $x^j\in N, ~j=0,1,2,\dots$. Hence we have
    \begin{align*}
        \|x^j - \bar x\| &= \|g(x^{j-1}) - g(\bar x)\| \\
        &\leq L\|x^{j-1} - \bar x\| \\
        & = L\|g(x^{j-2}) -g(\bar x)\| \\
        &\leq L^2\|x^{j-2} - \bar x\| \\
        &\vdots \\
        &\leq L^j\|x^0 - \bar x\|.
    \end{align*}
    It follows that as $j\to \infty, ~ \|x^0 - \bar x\| \to 0$.
    
    % Part (b)
    \item Suppose there exists a norm $\|\cdot\|$ such that $g(x)$ is Lipschitz continuous with constant $L<1$ in a neighborhood
    of $\bar x$. Fix $\epsilon>0$ small enough so that $\bar x + y \in N$ for any $\|y\|<\epsilon$, where $N$ is the neighborhood around
    $\bar x$ in which $g$ is Lipschitz continuous. Taylor expanding $g(\bar x+y)$ about $\bar x$ gives
    \[
    g(\bar x + y) = g(\bar x) + g'(\bar x)y + O(\|y\|^2) =  g(\bar x) + g'(\bar x)y + v
    \]
    for some $v$ with $\|v\| \leq C\|y\|^2$, some $C\in \reals$. From this it follows that
    \[
    g'(\bar x)y = -v + g(\bar x +y) -g(\bar x),
    \]
    and so
    \begin{align*}
    \|g'(\bar x)y\| &= \|-v + g(\bar x +y) -g(\bar x)\| \\
    &\leq \|v\| + \|g(\bar x+y) - g(\bar x)\| \\
    &\leq C\|y\|^2 + L\|y\|.
    \end{align*}
    Since this holds for any $\|y\|<\epsilon$, choose $y$ to be the unit-length eigenvector corresponding to the largest (in magnitude)
    eigenvalue of $g'(\bar x),~ \lambda$, scaled by $\frac{\epsilon}{2}$, i.e. $\rho(g'(\bar x))=\|\lambda\|$. Then the above becomes
    \begin{align*}
    \|g'(\bar x)y\| = |\lambda|\|y\| &\leq C\|y\|^2 + L\|y\| \\
    \Rightarrow |\lambda| &\leq C\|y\| + L  \\
    &= C\frac{\epsilon}{2} + L.
    \end{align*}
    As $\epsilon$ can be made arbitrarily small and $L<1$, we conclude that $\rho(g'(\bar x))=|\lambda| < 1$.
    
    % Part (c)
    \item Applied to the test problem $u' = \lambda u$ this method yields
    $$ \hat U^0 =U^n + kf(U^n) =  U^n + k\lambda U^n = (1 + z)U^n,$$
    where $z = k\lambda$. Next 
    \begin{align*}
     \hat U^1 &= U^n + kf(\hat U^0) = U^n + k\lambda(1+z)U^n = (1 + z + z^2)U^n, \\
     \hat U^2 &= U^n + kf(\hat U^1) = U^n + k\lambda(1+z+z^2)U^n = (1 + z + z^2 + z^3)U^n \\
     &\vdots \\
     \hat U^N &= (1 + z + z^2 + \dots + z^{N+1})U^n.
    \end{align*}
    Thus $R(z) = 1 + z + z^2 + \dots + z^{N+1}$.
    
    \begin{figure}[!ht]
    \centering
    \includegraphics[width=.4\textwidth]{S2}\quad
    \includegraphics[width=.4\textwidth]{S5}\\
    \includegraphics[width=.4\textwidth]{S10}\quad
    \includegraphics[width=.4\textwidth]{S20}
    \caption{Top left: $S_2$, Top right: $S_5$, Bottom left: $S_{10}$, Bottom right: $S_{20}$}
    \label{fig:stab}
    \end{figure}
    
    Using {\tt plotS.m} as suggested, we generate the following plots showing the stability regions
    $S_N$ for $N=2,~5,~10,~20$ (Figure \ref{fig:stab}). Notice that the stability region does not grow much in size as $N$ is increased.
    
    
    
    % Part (d)
    \item Interpreting the method presented in (c) as a fixed point iteration to solve the nonlinear equation
    \[
    \unp = U^n + kf(\unp)
    \]
    we take 
    \[
    g(x) = U^n + kf(x).
    \]
    
    If the result from $(a)$ is to hold and $g(x)$ is to converge to the fixed point $U^{n+1}$ as desired, the condition from (a) must be satisfied.
    If this is the case, then the result from (b) holds, namely 
    $$\rho(g'(U^{n+1})) = \rho((U^n + kf(u))' ) = \rho(kf'(u)) < 1.$$ 
    But 
    $\rho(kf'(u)) = k\rho(f'(u)),$ so this condition is equivalent to requiring that $|k\lambda|<1$ for all eigenvalues $\lambda$ of $f'(u)$. Notice
    that this requirement is slightly different from the result of (b) in that we now require that $ \rho(kf'(u)) < 1$ for all value $u$, not just 
    $U^{n+1}.$ This is because we modify the contraction map $g$ at each time step so that its fixed point becomes the next approximation we wish
    to compute. Therefore we must ensure that the entire family of contractions maps converge to their respective fixed points. If this condition is
    not satisfied then the result in (b) does not hold, so neither does the result in (a) and the map cannot be expected to converge.
    
    % Part (e)
    \item Based on part (d) we expect that the stability region $S_N$ will approach the unit circle, i.e. the set on which $|\lambda k| = |z| < 1$
    as this is where we expect the method to converge. The plots in Figure \ref{fig:stab} support this expectation. As $N$ is increased the
    stability regions look more and more like the unit circle.
    
    \end{enumerate}
    
    
\end{homeworkProblem}

\pagebreak

% Problem 2
\begin{homeworkProblem}
   Use the Boundary Locus method to plot the stability region for the TR-BDF2
    method (8.6).   You can use the Matlab script \texttt{makeplotS.m}
    from the book website, or a Python script as illustrated in 
    {\tt \$AM586/codes/notebook3.ipynb}
    
    Observe that the method is A-stable and show that it is
    also L-stable.
    
    \vskip 1cm
    \textbf{Solution:} \\ 
    
    The TR-BDF2 method is given by
    \begin{align*}
        U^* &=U^n+\frac{k}{4}(f(U^n) + f(U^*)) \\
        U^{n+1} &= \frac{1}{3}(4U^* - U^n +kf(U^{n+1})).
    \end{align*}
    For the test problem, $u' = \lambda u$ we can derive the relations
    \[
    U^* = \left(\frac{1 + \frac{z}{4}}{1 - \frac{z}{4}} \right)U^n
    \]
    and
    \[
    U^{n+1} = \left( \frac{12 + 5z}{12-7z+z^2}\right)U^n,
    \]
    where $z = \lambda k$. Hence $R(z) =  \frac{12 + 5z}{12-7z+z^2}$. 
    
    \begin{figure}
    \centering
    \includegraphics[width=.65\textwidth]{TRBDF2}
    \caption{Stability region for TR-BDF2}
    \label{fig:TRBDF2}
    \end{figure}
    Using {\tt plotS.m} we plot the stability region for the TR-BDF2 method 
    in Figure \ref{fig:TRBDF2}. It is easy to see from this plot that the method is A-stable. To see that it is also L-stable, simply notice that
    \[
    \lim_{z\to \pm \infty} R(z) = \lim_{z\to \pm \infty} \frac{12 + 5z}{12-7z+z^2} = 0.
    \]
    
    
\end{homeworkProblem}

\pagebreak

% Problem 3
\begin{homeworkProblem}
    The goal of this problem is to write a very simple adaptive time step ODE
    solver based on the Bogacki--Shampine Runge--Kutta method.  This is a
    third-order accurate 4-stage method that also produces a second-order
    accurate approximation each time step that can be used for error estimation. 
    (This is the method used in the Matlab routine \texttt{ode23}.)
    
    Note that it appears to require 4 evaluations of $f$ each time step, but the
    last one needed in one step can be re-used in the next step.  You should
    take advantage of this to reduce it to 3 new $f$-evaluations each step.
    
    In lecture, I discussed the method (5.42) in which $U^{n+1}$ is used for the
    next time step and is second-order accurate, while $\hat U^{n+1}$ is 
    ``first-order accurate'', which means the 1-step error is $\bigo(k^2)$, and
    this is what is approximated by the difference $|U^{n+1}-\hat U^{n+1}|$.
    
    For the Bogacki--Shampine method, the corresponding difference $|U^{n+1}-\hat
    U^{n+1}|$ will be approximately equal to the 1-step error of the second order
    method, so we expect it to have the form $k_n\tau^n \approx C_n k_n^3$, where
    $k_n$ is the time
    step just used in the $n$th step, and $C_n$ is a constant that will depend
    on how smooth the solution is near time $t_n$.  After each time step, we
    can estimate this by
    \begin{equation}\label{Cn}
    C_n \approx |U^{n+1}-\hat U^{n+1}|^{1/3}.
    \end{equation} 
    If we were simply using the second order method and trying to achieve an
    absolute error less than some $\epsilon$ over the time interval $t_0 \leq t
    \leq T$, then we would want to choose our time steps so that 
    \[
    |\tau^n| \leq \epsilon / (T-t_0)
    \]
    Then if we assume the method is behaving stably and we estimate the global
    error at time $T$ by the sum of the one-step errors, this is roughly
    \[
    \sum_m k_m |\tau^m| \leq \left(\sum k_m\right) \frac{\epsilon}{T-t_0} =
    \epsilon.
    \]
    Convince yourself that this suggests choosing the next time step as
    \begin{equation}\label{kn}
    \left(\frac{\epsilon}{C_n(T-t_0)}\right)^{1/2}.
    \end{equation} 
    
    Note that we have already taken a step with time step $k_n$ in order to
    approximate $C_n$, so the simplest approach is just to use \eqn{kn}
    to define our {\em next} time step $k_{n+1}$, with $C_n$ estimated
    from \eqn{Cn}.  (A more sophisticated method would go back and
    re-take a smaller step if the estimate indicates we took a step
    that was too large.)
    
    Note that the error estimate is based on the second-order method, but the
    value we take for our next step is $U^{n+1}$, which we hope is even more
    accurate.
    
    \newpage
    Implement this idea, with the following steps:
    
    \begin{itemize} 
    \item First implement a function that takes a single time step with the B--S
    method and confirm that it is working, e.g. if you apply it to $u' = -u$
    for a single step of size $k$ the errors behave as expected when you reduce
    $k$.  
    
    \item Then implement a fixed time step method based on this and verify that
    the method is third-order accurate.
    
    \item Implement an adaptive time version and test it on various 
    problems where you know the solution.
    
    \end{itemize} 
    
    {\bf Note:} You will need to choose a time step for the first step.  To keep
    it simple, just use $k_0 = 10^{-4}$.   You should also set some maximum
    step size that's allowed since there may be times when the error estimate
    happens to be very small.  Take this to be $k_{max} = 10 \epsilon^{1/3}$.
    
    
    Once your code is working, try it on problems of the form
    
    \begin{equation}\label{fuv}
    u'(t) = \lambda (u - v(t)) + v'(t)
    \end{equation} 
    where $v(t)$ is a desired exact solution and the initial data is chosen as 
    $u(0) = v(0)$.
    
    In particular, try the following.  Produce some sample plots of the
    solutions and also plots of the error and step size used as functions of
    $t$, as illustrated in the figures below.
    
    \begin{enumerate} 
    \item  Exact solution
    \[
    v(t) = \cos(t) 
    \]
    for $0\leq t \leq 10$ and $\lambda = -1$. Try different tolerances
    $\epsilon$ in the range $10^{-2}$ to $10^{-10}$.
    
    \item
    Try the above problem with more negative $\lambda$, e.g. $\lambda=-100$.
    What happens if $\epsilon$ is large enough that $k_{acc} > k_{stab}$?  Does
    the method go unstable?
    
    
    \item  Exact solution
    \[
    v(t) = \cos(t) + \exp\left(-80(t-2)^2\right),
    \]
    for $0\leq t \leq 4$ and $\lambda = -1$. The Gaussian gives a region where the
    solution is much more rapidly varying and smaller time steps are required.
    \end{enumerate}

    \vskip 1cm
    \textbf{Solution:} \\
    
    See attached Julia notebook for verification that the single step and fixed time step methods achieve proper orders of accuracy,
    method implementations, and extra plots.
    
    \begin{enumerate}
    
        % Part (a)
        \item First we apply the adaptive time-stepping method to the differential equation above with exact solution
        \[
        \nu(t) = cos(t).
        \]
        
        
        \begin{figure}[h]
        \centering
        \begin{tabular}{|l|l|l|}
         \hline
         $\epsilon$ & Absolute Error & Steps taken\\ \hline
         $10^{-2}$ & 4.2e-3 & 22\\ \hline
         $10^{-3}$ & 1.2e-3 & 94\\ \hline
         $10^{-4}$ & 1.1e-5 & 330\\ \hline
         $10^{-5}$ & 5.3e-7 & 1065\\ \hline
         $10^{-6}$ & 4.6e-8 & 6403\\ \hline
         $10^{-7}$ & 1.2e-9 & 10777\\ \hline
         $10^{-8}$ & 1.0e-10 & 34110\\ \hline
         $10^{-9}$ & 2.2e-12 & 1.1e6\\ \hline
         $10^{-10}$ & 9.0e-10 & 2.1e6\\ \hline
         \hline
        \end{tabular}
        \caption{Problem 3(a)}
        \label{tab:parta}
        \end{figure}
        
        The results are summarized in Figure \ref{tab:parta}. As the error tolerance is decreased, the number of time steps taken 
        increases rapidly. The method does a fairly good job of staying within the error tolerances provided. The error is at worst
        the same order of magnitude as the specified tolerances.
        
        \begin{figure}[h!]
        \centering
        \includegraphics[width=0.6\textwidth]{3aplot} \\
        \includegraphics[width=0.6\textwidth]{3aerror}
        \caption{Plot, error, and time step size for 3(a), $\epsilon=10^{-3}$}
        \label{fig:parta}
        \end{figure}
        
        Figure \ref{fig:parta} shows the numerical approximation, absolute error, and time step size for this problem when 
        $\epsilon=10^{-3}.$ We see that the size of the time step is lengthened around the areas where the exact solution is fairly
        flat, where its derivative is zero. This is the type of behavior we expect from an adaptive time stepping method--larger
        time steps when the solution is changing slowly and smaller ones when it is rapidly varying.
        
        % Part (b)
        \item Next we test the method on a more stiff version of the problem above, taking $\lambda =-100$ rather than $-1$.
        
        \begin{figure}[ht]
        \centering
        \begin{tabular}{|l|l|l|}
         \hline
         $\epsilon$ & Absolute Error & Steps taken\\ \hline
            $10^{-2}$ & 1.8e-3 & 710 \\ \hline
            $10^{-3}$ & 3.2e-6 & 1038 \\ \hline
            $10^{-4}$ & 3.4e-7 & 1968 \\ \hline
            $10^{-5}$ & 3.9e-8 & 4085 \\ \hline
            $10^{-6}$ & 4.3e-9 & 8623 \\ \hline
            $10^{-7}$ & 5.1-10 & 18678 \\ \hline
            $10^{-8}$ & 7.3-11 & 44637 \\ \hline
            $10^{-9}$ & 2.8e-7 & 1.2e6 \\ \hline
            $10^{-10}$ & 3.6e-7 & 1.0e6 \\ \hline
         \hline
        \end{tabular}
        \caption{Problem 3(b)}
        \label{tab:partb}
        \end{figure}
        
        \begin{figure}[h!]
        \centering
        \includegraphics[width=0.4\textwidth]{3bplotgood} \qquad
        \includegraphics[width=0.4\textwidth]{3bplotbad} \\
        \includegraphics[width=0.4\textwidth]{3berrorgood} \qquad
        \includegraphics[width=0.4\textwidth]{3berrorbad}
        \caption{Plot, error, and time step size for 3(b) (Left: $\epsilon=10^{-3}$, Right: $\epsilon=10^{-1}$)}
        \label{fig:partb}
        \end{figure}
        
        Figure (\ref{tab:partb}) summarizes the results over the same range of $\epsilon$ as was used in part (a). In response to
        the stiffness of the problem, more time steps are needed to stay within each given error tolerance. This particular
        method is unable to keep the absolute error under $10^{-9}$ and $10^{-10}$ when it is asked to, although it achieves
        these levels of accuracy when $\epsilon$ is as large as $10^{-6}$ and $10^{-7}$, respectively.
        
        When $\epsilon$ is made large enough that $k_{acc} > k_{stab}$, the method is allowed to take a large enough time step that
        the approximate solution blows up temporarily until the time step is reduced appropriately. Then the accuracy improves
        substantially. This phenomenon is demonstrated on the right in Figure (\ref{fig:partb}). When $\epsilon=10^{-1}$ 
        the first few approximations are much too large, but they quickly settle down as the method decreases the step size.
        The error then oscillates just below $10^{-1}$ until the final time is reached. If $\epsilon$ is taken to be too large
        then $k_max$ becomes larger than $T-t_0$ and so the method takes only two steps (an initial step of $10^{-4}$ then
        another of length $T-t_0 - 10^{-4}$) to obtain an approximation at the final time.
        \pagebreak
        
        % Part (c)
        \item Finally we apply the Bogacki-Shampine method to the above differential equation with exact solution
        \[
        v(t) = \cos(t) + \exp\left(-80(t-2)^2\right).
        \]
        
        \begin{figure}[ht]
        \centering
        \begin{tabular}{|l|l|l|l|l|}
         \hline
         & \multicolumn{2}{|c|}{$k_{max}=10\epsilon^{\frac{1}{3}}$} & 
         \multicolumn{2}{|c|}{$k_{max}=\epsilon^{\frac{1}{3}}$} \\ \hline
         $\epsilon$ & Absolute Error & Steps taken & Absolute Error & Steps taken\\ \hline
            $10^{-2}$ & 4.0e-2 & 10 & 6.7e-4 & 63\\ \hline
            $10^{-3}$ & 3.0e-3 & 164 & 1.5e-3 & 158\\ \hline
            $10^{-4}$ & 2.2e-5 & 564 & 2.7e-6 & 586\\ \hline
            $10^{-5}$ & 3.3e-6 & 1801 & 1.3e-7 & 1827\\ \hline
            $10^{-6}$ & 2.5e-7 & 5739 & 3.6e-7 & 5715\\ \hline
            $10^{-7}$ & 9.7e-10 & 18190 & 4.2e-10 & 18207\\ \hline
            $10^{-8}$ & 6.5e-11 & 57513 & 2.3e-11 & 57563\\ \hline
            $10^{-9}$ & 1.8e-12 & 1.8e6 & 1.1e-12 & 1.8e6\\ \hline
            $10^{-10}$ & 5.2e-8 & 94885 & 7.8e-11 & 1.2e6\\ \hline
         \hline
        \end{tabular}
        \caption{Problem 3(c)}
        \label{tab:partc}
        \end{figure}
        
        \begin{figure}[h]
        \centering
        \includegraphics[width=0.4\textwidth]{3cplot} \qquad
        \includegraphics[width=0.4\textwidth]{3cplotsmallkmax} \\
        \includegraphics[width=0.4\textwidth]{3cerror} \qquad
        \includegraphics[width=0.4\textwidth]{3cerrorsmallkmax}
        \caption{Plot, error, and time step size for 3(c) (Left: $\epsilon = 10^{-3},~ k_{max} = 10\epsilon^{\frac{1}{3}}$
        , Right: $\epsilon = 10^{-2},~ k_{max} = \epsilon^{\frac{1}{3}}$)}
        \label{fig:partc}
        \end{figure}
        
        We found that we needed to take either $k_{max}$ or $\epsilon$ to be slightly smaller than was specified in the 
        assignment to obtain plots that look similar to those on the handout. Taking $\epsilon=10^{-3}$ rather than $10^{-2}$ gives
        the approximation, error, and time step sizes shown on the left of Figure (\ref{fig:partc}) and taking
        $k_{max}=\epsilon^{\frac{1}{3}}$ instead of $10(\epsilon)^\frac{1}{3}$ gives those on the right. Both approaches show
        $k$ being decreased at the time when the extra gaussian term causes an abrupt change in the solution behavior. 
        Notice how large the second time step is for the larger choice of $k_{max}$. In fact, if we choose $\epsilon = 10^{-2}$
        along with the larger $k_{max}$, we get $k_{max} = 10(\epsilon)^\frac{1}{3} \approx2.15$ so when the second time step
        is taken to be $k_{max}$ (as it was observed to do in every instance tested), the approximate solution is computed
        at a point past the rapidly varying part of the exact solution. The adaptive method takes too large a time step and misses
        the "interesting" part of the solution entirely. It is for this reason that we compare the method's performance using
        two different formulas for $k_{max}$. Figure (\ref{tab:partc}) shows the results of both versions of the B-S method.
        Notice that for the more restrictive definition of $k_{max}$ the method requires roughly the same number of time steps
        to reach the final time and achieves slightly better accuracy for every trial. If given a choice between the two, we would
        use the method using the more constrained maximum time step (they both run almost instantaneously in any case).
        
    \end{enumerate}
    

\end{homeworkProblem}




\end{document}


