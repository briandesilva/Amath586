\documentclass{article}

\usepackage{fancyhdr}
\usepackage{extramarks}
\usepackage{amsmath}
\usepackage{amsthm}
\usepackage{amsfonts}
\usepackage{amssymb}
\usepackage{graphicx}
\usepackage{caption,subcaption}

%
% Basic Document Settings
%

\topmargin=-0.45in
\evensidemargin=0in
\oddsidemargin=0in
\textwidth=6.5in
\textheight=9.0in
\headsep=0.25in

\linespread{1.1}

\pagestyle{fancy}
\lhead{\hmwkAuthorName}
\chead{\hmwkClass:\ \hmwkTitle}
\rhead{\firstxmark}
\lfoot{\lastxmark}
\cfoot{\thepage}

\renewcommand\headrulewidth{0.4pt}
\renewcommand\footrulewidth{0.4pt}

\setlength\parindent{0pt}

%
% Create Problem Sections
%

\newcommand{\enterProblemHeader}[1]{
    \nobreak\extramarks{}{Problem \arabic{#1} continued on next page\ldots}\nobreak{}
    \nobreak\extramarks{Problem \arabic{#1} (continued)}{Problem \arabic{#1} continued on next page\ldots}\nobreak{}
}

\newcommand{\exitProblemHeader}[1]{
    \nobreak\extramarks{Problem \arabic{#1} (continued)}{Problem \arabic{#1} continued on next page\ldots}\nobreak{}
    \stepcounter{#1}
    \nobreak\extramarks{Problem \arabic{#1}}{}\nobreak{}
}

\setcounter{secnumdepth}{0}
\newcounter{partCounter}
\newcounter{homeworkProblemCounter}
\setcounter{homeworkProblemCounter}{1}
\nobreak\extramarks{Problem \arabic{homeworkProblemCounter}}{}\nobreak{}

%
% Homework Problem Environment
%
% This environment takes an optional argument. When given, it will adjust the
% problem counter. This is useful for when the problems given for your
% assignment aren't sequential. See the last 3 problems of this template for an
% example.
%
\newenvironment{homeworkProblem}[1][-1]{
    \ifnum#1>0
        \setcounter{homeworkProblemCounter}{#1}
    \fi
    \section{Problem \arabic{homeworkProblemCounter}}
    \setcounter{partCounter}{1}
    \enterProblemHeader{homeworkProblemCounter}
}{
    \exitProblemHeader{homeworkProblemCounter}
}

%
% Homework Details
%   - Title
%   - Due date
%   - Class
%   - Section/Time
%   - Instructor
%   - Author
%

\newcommand{\hmwkTitle}{Homework 1}
\newcommand{\hmwkDueDate}{April 9, 2015}
\newcommand{\hmwkClass}{Amath 586}
\newcommand{\hmwkAuthorName}{Brian de Silva}

%
% Title Page
%

\title{
    \vspace{2in}
    \textmd{\textbf{\hmwkClass:\ \hmwkTitle}}\\
    \normalsize\vspace{0.1in}\small{Due\ on\ \hmwkDueDate\ }\\
    \vspace{3in}
}

\author{\textbf{\hmwkAuthorName}}
\date{}

\renewcommand{\part}[1]{\textbf{\large Part \Alph{partCounter}}\stepcounter{partCounter}\\}

%
% Various Helper Commands
%

% Useful for algorithms
\newcommand{\alg}[1]{\textsc{\bfseries \footnotesize #1}}

% For derivatives
\newcommand{\deriv}[1]{\frac{\mathrm{d}}{\mathrm{d}x} (#1)}

% For partial derivatives
\newcommand{\pd}[2]{\frac{\partial}{\partial #1} (#2)}

\newcommand{\pdd}[2]{\frac{\partial #1}{\partial #2}}

% Integral dx
\newcommand{\dx}{\mathrm{d}x}

% Alias for the Solution section header
\newcommand{\solution}{\textbf{\large Solution}}

% Probability commands: Expectation, Variance, Covariance, Bias
\newcommand{\E}{\mathrm{E}}
\newcommand{\Var}{\mathrm{Var}}
\newcommand{\Cov}{\mathrm{Cov}}
\newcommand{\Bias}{\mathrm{Bias}}

% Some useful macros
\input{./macros.tex}

\begin{document}

\maketitle

\pagebreak

% Problem 1
\begin{homeworkProblem}
    Prove that the ODE 
    \[
    u'(t) = \frac 1 {t^2 + u(t)^2}, \quad \mbox{for}~t \geq 1
    \]
    has a unique solution for all time from any initial value $u(1)=\eta$.
    
    \vskip 1cm
    \textbf{Solution:} \\
   Let \( f(u,t) = u'(t) \), $D~=~\reals \times [1,\infty),$ and observe that \( 0\leq f(u,t) \leq 1\) 
   for \(t\geq 1\) \\
   as \(t^2 + u(t)^2 > 1 \) for \( (u,t)\in D\). Also notice that for $(u,t)\in D$
   \[\pdd{f}{u} = f_u = \frac{-2u}{(t^2 + u(t)^2)^2}, \]
   so 
   \[\lvert f_u\rvert = \frac{2|u|}{(t^2 + u(t)^2)^2} \leq \frac{2|u|}{(1 + u(t)^2)^2} \leq\frac{3\sqrt{3}}{8}. \]
   The bound above can be obtained by locating the local maxima and minima of $g(u) = \frac{2u}{(1+u^2)^2}$.
   
   Since $f$ is differentiable with respect to $u$ for any $(u,t) \in D$, and $f_u$ is bounded for 
   $(u,t) \in D$ we can apply the result from Section 5.2.1 which states that if $f$ is Lipschitz continuous over
   a domain $D$, then there is a unique solution to the initial value problem above at least up to time 
   $T^* =min(t_1,t_0 + a/S)$, where
   \[
        S = \max_{(u,t)\in D} |f(u,t)| \leq 1,
   \]
    and $a,~t_0,$ and $t_1$ appear in the definition of $D$ as 
    \[
        D~=~ \{(u,t) :~ |u-\eta|\leq a, ~t_0\leq t \leq t_1 \}
    \]
    (they specify the domain over which $f$ is Lipschitz continuous). But the calculations above show that
    $t_0 = 1$ and that $a$ and $t_1$ can be taken to be arbitrarily large. It follows that $T^* = \infty$, 
    i.e. that there exists a unique solution to the provided initial value problem for any $\eta$, and for all 
    $t\geq 1$.
\end{homeworkProblem}

\pagebreak

% Problem 2
\begin{homeworkProblem}
   Consider the system of ODEs
    \begin{equation*}
    \begin{split}
    u_1' &= 3u_1 + 4u_2,\\
    u_2' &= 5u_1 - 6u_2.\\
    \end{split}
    \end{equation*}
    Determine the best possible Lipschitz constant for this system in the max-norm
    $\|\cdot\|_\infty$ and the 1-norm $\|\cdot\|_1$. (See Appendix A.3.)
    
    \vskip 1cm
    \textbf{Solution:} \\
    This system can be written as \(\textbf{u}^\prime = A\textbf{u}\) where
    \begin{equation*}
        A = 
            \bcm
                3 & 4 \\
                5 & -6
            \ecm , ~\\ 
        \textbf{u} = 
            \bcm
                u_1 \\
                u_2
            \ecm.
    \end{equation*}
    Recall that for any vector norm \(\|\cdot\|\) inducing a submultiplicative matrix norm we have
    
    \begin{equation*}
    \|A\textbf{x} - A\textbf{y}\| = \|A(\textbf{x}-\textbf{y})\| \leq \|A\| ~ \|\textbf{x}-\textbf{y}\|
    \end{equation*}
    for any \(\textbf{x},\textbf{y}\). The (matrix) 1-norm and $\infty$-norm are just the maximum absolute column
    and row sums, respectively, so $\|A\|_1 = 4 + 6 = 10$ and $\|A\|_\infty = 5+6=11$.
    Since the 1-norm and $\infty$-norm induce submultipliative matrix norms,
    \begin{equation*}
        \|A\textbf{u}-A\textbf{u}^*\|_1 \leq \|A\|_1 ~ \|\textbf{u}-\textbf{u}^*\|_1 = 10\|\textbf{u}-\textbf{u}^*\|_1,
    \end{equation*}
    \begin{equation*}
         \|A\textbf{u}-A\textbf{u}^*\|_\infty \leq \|A\|_\infty ~ \|\textbf{u}-\textbf{u}^*\|_\infty = 11\|\textbf{u}-\textbf{u}^*\|_\infty.
    \end{equation*}
    Thus 10 and 11 are Lipschitz constants for the system with respect to the 1 and infinity norms, respectively.
    To see that they are the best possible Lipshitz constants, notice that equality is attained in the above
    two equations for 
    \begin{equation*}
        \textbf{u}-\textbf{u}^* = \bcm 0 \\ 1 \ecm, ~ \ \textbf{u}-\textbf{u}^* = \bcm 1 \\ -1 \ecm
    \end{equation*}
    which are unit vectors with respect to the 1 and infinity norms, respectively.
    
\end{homeworkProblem}

\pagebreak

% Problem 3
\begin{homeworkProblem}
    The initial value problem
    \[
    v''(t) = -4v(t), \qquad v(0) = v_0,\quad v'(0) = v_0'
    \]
    has the solution $v(t) = v_0\cos(2t) + \half v_0' \sin(2t)$.  Determine this
    solution by rewriting the ODE as a first order system $u' = Au$ so that
    $u(t) = e^{At}u(0)$ and then computing the matrix exponential using (D.30)
    in Appendix D.

    \vskip 1cm
    \textbf{Solution:} \\
    Letting $x(t) = v(t)$ and $ y(t) = v'(t)$, we can rewrite the initial value problem as the following system
    \begin{align*}
        x' &= y, \\
        y' &= -4x
    \end{align*}
    with initial conditions $x(0) = v_0, y(0)=v'_0$. In turn, we can represent this system as the matrix equation
    $\textbf{u}'=A\textbf{u}$, with
    \[
    \textbf{u} = \bcm x \\ y \ecm,
    A = \bcm
            0 & 1 \\
            -4 & 0
        \ecm.
    \]
    Then the solution to the system is given by 
    \[
    \textbf{u}(t) = e^{At}\textbf{u}(0).
    \]
    
    In order to compute the matrix exponential $e^{At}$ we will need the Jordan decomposition of $A$. \\
    To this end $det(A-\lambda  I) = \lambda^2 + 4$, so the eigenvalues of $A$ are simply $\pm 2i$. 
    It can easily be shown that the eigenvectors of $A$ are scalar multiples of 
    \[
    v_1 = \bcm i \\ 2 \ecm, v_2 = \bcm -i \\ 2 \ecm.
    \]
    Letting $R = [v_1,~ v_2]$ we see that $A = R\Lambda R^{-1}$, where 
    \[
    \Lambda =
    \bcm
        -2i & 0 \\
        0 & 2i
    \ecm.
    \]
    Hence 
    \begin{align*}
        e^{At} &= e^{R\Lambda R^{-1}} \\
        &= Re^{\Lambda t}R^{-1} \\
        &= \bcm i & -i \\ 2 & 2 \ecm \bcm e^{-2it} & 0 \\ 0 & e^{2it} \ecm
        \bcm \frac{-i}{2} & \frac{1}{4} \\ \frac{i}{2} & \frac{1}{4}  \ecm \\
        &= \bcm \cos(2t) & \frac{1}{2}\sin(2t) \\ -2\sin(2t) & \cos(2t) \ecm.
    \end{align*}
    
    Therefore, 
    \[
    \textbf{u} = \bcm x \\ y \ecm = \bcm v \\ v' \ecm  = e^{At}\textbf{u}(0) = 
    \bcm \cos(2t) & \frac{1}{2}\sin(2t) \\ -2\sin(2t) & \cos(2t) \ecm \bcm v_0 \\ v'_0\ecm = 
    \bcm v_0\cos(2t) + \frac{v'_0}{2}\sin(2t) \\ -2v_0\sin(2t) + v'_0\cos(2t) \ecm,
    \]
    from which we get $v(t) = v_0\cos(2t) + \frac{v'_0}{2}\sin(2t)$.
    
    
    
    % \part

    % Find the least squares estimator for \(\hat{\beta_1}\) for the slope
    % \(\beta_1\).
    % \\

    % \solution

    % To find the least squares estimator, we should minimize our Residual Sum
    % of Squares, RSS:

    % \[
    %     \begin{split}
    %         RSS &= \sum_{i = 1}^{n} {(Y_i - \hat{Y_i})}^2
    %         \\
    %         &= \sum_{i = 1}^{n} {(Y_i - \hat{\beta_1} x_i)}^2
    %     \end{split}
    % \]

    % By taking the partial derivative in respect to \(\hat{\beta_1}\), we get:

    % \[
    %     \pd{
    %         \hat{\beta_1}
    %     }{RSS}
    %     = -2 \sum_{i = 1}^{n} {x_i (Y_i - \hat{\beta_1} x_i)}
    %     = 0
    % \]

    % This gives us:

    % \[
    %     \begin{split}
    %         \sum_{i = 1}^{n} {x_i (Y_i - \hat{\beta_1} x_i)}
    %         &= \sum_{i = 1}^{n} {x_i Y_i} - \sum_{i = 1}^{n} \hat{\beta_1} x_i^2
    %         \\
    %         &= \sum_{i = 1}^{n} {x_i Y_i} - \hat{\beta_1}\sum_{i = 1}^{n} x_i^2
    %     \end{split}
    % \]

    % Solving for \(\hat{\beta_1}\) gives the final estimator for \(\beta_1\):

    % \[
    %     \begin{split}
    %         \hat{\beta_1}
    %         &= \frac{
    %             \sum {x_i Y_i}
    %         }{
    %             \sum x_i^2
    %         }
    %     \end{split}
    % \]

    % \pagebreak

    % \part

    % Calculate the bias and the variance for the estimated slope
    % \(\hat{\beta_1}\).
    % \\

    % \solution

    % For the bias, we need to calculate the expected value
    % \(\E[\hat{\beta_1}]\):

    % \[
    %     \begin{split}
    %         \E[\hat{\beta_1}]
    %         &= \E \left[ \frac{
    %             \sum {x_i Y_i}
    %         }{
    %             \sum x_i^2
    %         }\right]
    %         \\
    %         &= \frac{
    %             \sum {x_i \E[Y_i]}
    %         }{
    %             \sum x_i^2
    %         }
    %         \\
    %         &= \frac{
    %             \sum {x_i (\beta_1 x_i)}
    %         }{
    %             \sum x_i^2
    %         }
    %         \\
    %         &= \frac{
    %             \sum {x_i^2 \beta_1}
    %         }{
    %             \sum x_i^2
    %         }
    %         \\
    %         &= \beta_1 \frac{
    %             \sum {x_i^2 \beta_1}
    %         }{
    %             \sum x_i^2
    %         }
    %         \\
    %         &= \beta_1
    %     \end{split}
    % \]

    % Thus since our estimator's expected value is \(\beta_1\), we can conclude
    % that the bias of our estimator is 0.
    % \\

    % For the variance:

    % \[
    %     \begin{split}
    %         \Var[\hat{\beta_1}]
    %         &= \Var \left[ \frac{
    %             \sum {x_i Y_i}
    %         }{
    %             \sum x_i^2
    %         }\right]
    %         \\
    %         &=
    %         \frac{
    %             \sum {x_i^2}
    %         }{
    %             \sum x_i^2 \sum x_i^2
    %         } \Var[Y_i]
    %         \\
    %         &=
    %         \frac{
    %             \sum {x_i^2}
    %         }{
    %             \sum x_i^2 \sum x_i^2
    %         } \Var[Y_i]
    %         \\
    %         &=
    %         \frac{
    %             1
    %         }{
    %             \sum x_i^2
    %         } \Var[Y_i]
    %         \\
    %         &=
    %         \frac{
    %             1
    %         }{
    %             \sum x_i^2
    %         } \sigma^2
    %         \\
    %         &=
    %         \frac{
    %             \sigma^2
    %         }{
    %             \sum x_i^2
    %         }
    %     \end{split}
    % \]

\end{homeworkProblem}

\pagebreak

% Problem 4
\begin{homeworkProblem}
    Compute the leading term in the local truncation error of the following
    methods:
    \begin{enumerate}
    \item the trapezoidal method (5.22),
    \item the 2-step Adams-Bashforth method,
    \item the Runge-Kutta method (5.32).
    \end{enumerate}

    \vspace{1cm}
    \textbf{Solution:} \\ 
    
    Throughout this problem I will use $u$ to denote $u(t_n)$, $f$ for $f(u(t_n))$, etc. when it does not add
    confusion to do so.
    The following Taylor expansions will be useful later in the problem:
    \begin{align}
        u(t+k) &= u(t) + ku'(t) + \frac{k^2}{2}u''(t) + \frac{k^3}{3!}u'''(t) + O(k^4) \label{eq:unp1} \\
        u(t-k) &= u(t) - ku'(t) + \frac{k^2}{2}u''(t) - \frac{k^3}{3!}u'''(t) + O(k^4) \label{eq:unm1} \\
        f(u(t+k)) &= u'(t) + ku''(t) + \frac{k^2}{2}u'''(t)+\frac{k^3}{3!}u^{(4)}(t) + O(k^4) \label{eq:fnp1} \\
        f(u(t-k)) &= u'(t) - ku''(t) + \frac{k^2}{2}u'''(t) - \frac{k^3}{3!}u^{(4)}(t) + O(k^4) \label{eq:fnm1}
    \end{align}
    
    \begin{enumerate}
    \item
    
    The local truncation error for the trapezoidal method (5.22) is given by
    \[
        \tau^n = \frac{u(t_n+k) - u(t_n)}{k} - \half (f(u(t_n)) + f(u(t_n+k))).
    \]
    
    Substituting (\ref{eq:unp1}) and (\ref{eq:fnp1}) for $u(t_n+k)$ and $f(u(t_n+k))$ and cancelling some terms yields
    
    \begin{align*}
        \tau^n &=\frac{ku' + \frac{k^2}{2}u'' + \frac{k^3}{3!}u''' + O(k^4)}{k} - 
        \half (2u' + ku'' + \frac{k^2}{2}u''' + O(k^3)) \\
        &= u' + \frac{k}{2}u'' + \frac{k^2}{3!}u'''  - (u' + \frac{k}{2}u'' + \frac{k^2}{4}u''') + O(k^3) \\
        &= \frac{-k^2}{12}u'''(t_n) + O(k^3).
    \end{align*}
    
    
    \item
    
    The local truncation error for the 2-step Adams-Bashforth method is given by
     
     \[
        \tau^n = \frac{u(t_n+k) - u(t_n)}{k} - \half(3f(u(t_n))-f(u(t_n-k))).
     \]
     
     \pagebreak
     
     Substituting (\ref{eq:unp1}) and (\ref{eq:fnm1}) for $u(t_n+k)$ and $f(u(t_n-k))$ and cancelling some
     terms gives
     \begin{align*}
        \tau^n &= \frac{ku' + \frac{k^2}{2}u'' + \frac{k^3}{3!}u''' + O(k^4)}{k} - 
        \half(2u' + ku'' - \frac{k^2}{2}u''') + O(k^3) \\
        &=u' + \frac{k}{2}u'' + \frac{k^2}{3!}u''' - 
        (u' + \frac{k}{2}u'' - \frac{k^2}{4}u''') +O(k^3)\\ 
        &=\frac{5}{12}k^2u'''(t_n) + O(k^3).
     \end{align*}
     
     \item
     
     The local truncation error for the 2-stage Runge-Kutta method (5.32) is given by
     \[
     \tau^n = \frac{u(t_n+k) - u(t_n)}{k} -f\left(u(t_n) + \frac{k}{2}f(u(t_n),t_n),~t_n+\frac{k}{2}\right).
     \]
     In this problem I use $f$ to denote $f(u(t),t)$, $f_t$ to denote $\pdd{f}{t}(u(t),t)$, etc. when
     doing so does not add ambiguity. We need to use slightly different Taylor series expansions than before 
     to find the leading term of the local truncation error for this method. First, we have
     \begin{align*}
        u''(t) &= \frac{d}{dt}(f(u,t))= f_t + ff_u, \\
        u'''(t) &= \frac{d}{dt}(u''(t))\\
        &= \frac{d}{dt}(f_t + ff_u)\\ 
        &=f_{tt} + 2ff_{ut} + f_{uu}f^2 + (f_u)^2f + f_uf_t.
     \end{align*}
     
     Using these and the multidimensional Taylor theorem we obtain
     \begin{align*}
         u(t+k) &= u + ku' + \frac{k^2}{2}u'' + \frac{k^3}{3!}u''' + O(k^4) \\
         &= u(t) + kf + \frac{k^2}{2}(f_t + ff_u) \\
         &+ \frac{k^3}{3!}(f_{tt} + 2ff_{ut} + f_{uu}f^2 + (f_u)^2f 
         + f_uf_t) + O(k^4). \\
        f\left(u(t) + \frac{k}{2}f(u(t),t),~t+\frac{k}{2}\right) &= \\
        &f + \frac{k}{2}ff_u + \frac{k}{2}f_t +  \half(\frac{k}{2}f)^2f_{uu}+\frac{k^2}{8}f_{tt} +
        \frac{k}{2}(\frac{k}{2}f)f_{ut} + O(k^3).
     \end{align*}
     
     \pagebreak
     
     Substituting the above Taylor expansions into the expression for the local truncation error and 
     cancelling some terms gives
     
     \begin{align*}
        \tau^n &= \frac{kf + \frac{k^2}{2}(f_t + ff_u) + \frac{k^3}{3!}(f_{tt} + 2ff_{ut} + f_{uu}f^2 + (f_u)^2f 
         + f_uf_t)}{k}\\
         &- (f + \frac{k}{2}ff_u + \frac{k}{2}f_t +  \half(\frac{k}{2}f)^2f_{uu}+\frac{k^2}{8}f_{tt} +
        \frac{k}{2}(\frac{k}{2}f)f_{ut}) + O(k^3) \\
        &= f + \frac{k}{2}(f_t + ff_u) + \frac{k^2}{3!}(f_{tt} + 2ff_{ut} + f_{uu}f^2 + (f_u)^2f 
         + f_uf_t) \\
         &- (f + \frac{k}{2}(ff_u + f_t) + \frac{k^2}{8}(f^2f_{uu}+f_{tt} +2ff_{ut})) + O(k^3) \\ 
         &= \frac{k^2}{6}(\frac{1}{4}f_{tt} + \frac{1}{2}ff_{ut} + \frac{1}{4}f{uu}f^2 +(f_u)^2f +
         f_uf_t)+O(k^3).
     \end{align*}
    \end{enumerate}
    
\end{homeworkProblem}


% Problem 5
\begin{homeworkProblem}
    Determine the coefficients $\beta_0,~\beta_1,~\beta_2$ for the third
    order, 2-step Adams-Moulton method.  Do this in two different ways:
    \begin{enumerate} 
     \item Using the expression for the local truncation error in Section 5.9.1,
     \item Using the relation
     \[
     u(t_{n+2}) = u(t_{n+1}) + \int_{t_{n+1}}^{t_{n+2}}\,f(u(s))\,ds.
     \]
     Interpolate  a quadratic polynomial $p(t)$ through the three values
     $f(U^n),~f(U^{n+1})$ and $f(U^{n+2})$ and then integrate this polynomial
     exactly to obtain the formula.  The coefficients of the polynomial will
     depend on the three values $f(U^{n+j})$.   It's easiest to use the
     ``Newton form'' of the interpolating polynomial and consider the three
    times $t_n=-k$, $t_{n+1}=0$, and $t_{n+2}=k$ so that $p(t)$ has the form
    \[
    p(t) = A + B(t+k) + C(t+k)t
    \]
    where $A,~B$, and $C$ are the appropriate divided differences based on the
    data.  Then integrate from $0$ to $k$.   (The method has the same
    coefficients at any time, so this is valid.)
    \end{enumerate}
    
    \vspace{1cm}
    \textbf{Solution:}\\
    \begin{enumerate}
   \item 
    The expression for the local truncation error for a 2-step Adams-Moulton method given in section 5.9.1 is
    
    \begin{align*}
        \tau(t_{n+2}) &= \frac{1}{k}\left(\sum^2_{j=0}\alpha_j\right)u(t_n) + 
        \left(\sum^2_{j=0}(j\alpha_j-\beta_j) \right)u'(t_n) \\
        &+ k\left(\sum^2_{j=0}(j^2\alpha_j-j\beta_j) \right)u''(t_n)
        + k^2\left(\sum^2_{j=0}(j^3\alpha_j-j^2\beta_j) \right)u'''(t_n) + O(k^3).
    \end{align*}
    
    
    We already know $\alpha_0=0, \alpha_1=-1,\alpha_2=1$ (hence the $O(\frac{1}{k})$ term drops out),
    so we must choose $\beta_i,~ i=1,2,3$ so that the $O(1), ~O(k),$ and $O(k^2)$ terms vanish (if we try to 
    eliminate the $O(k^3)$ term we end up with an overdetermined system of equations which we cannot solve).
    Each sum gives an equation:
    
    \begin{align*}
        \left(\sum^2_{j=0}(j\alpha_j-\beta_j) \right)u'(t_n) &= (-\beta_0-\beta_1-\beta_2 + 1)u'(t_n) = 0
        &\Longrightarrow~ &\beta_0 &+ \beta_1&+\beta_2 &= 1 \\
        k\left(\sum^2_{j=0}(j^2\alpha_j-j\beta_j) \right)u''(t_n) &= k(-\beta_1-2\beta_2 + \frac{3}{2})u''(t_n)=0
        &\Longrightarrow~ & &~\beta_1 &+ 2\beta_2 &=\frac{3}{2} \\
        k^2\left(\sum^2_{j=0}(j^3\alpha_j-j^2\beta_j) \right)u'''(t_n) &= k^2(-\half\beta_1 - 2\beta_2 +
        \frac{7}{6})u'''(t_n)
        =0 &\Longrightarrow~ & &~\half\beta_1 &+ 2\beta_2 &=\frac{7}{6}.
    \end{align*}
    Solving these equations simultaneously for $\beta_0, \beta_1$, and $\beta_2$ yields
    $\beta_0 = -\frac{1}{12},~ \beta_1 = \frac{2}{3},$ and $\beta_2 = \frac{5}{12}$, giving the 2-step 
    Adams-Moulton method:
    \[
    U^{n+2} = U_{n+1} + \frac{k}{12}(-f(U^n) + 8f(U^{n+1}) + 5f(U_{n+2})).
    \]
    
    \item
    Taking $t_n=-k,~t_{n+1}=0,$ and $t_{n+2}=k$, the ``Newton form'' (and using divided differences notation)
    of the quadratic polynomial interpolating
    $f(U^n),~f(U^{n+1})$ and $f(U^{n+2})$ is given by
    \[
    p(t) = [f(U^{n})] + [f(U^{n}),f(U^{n+1})](t+k) + [f(U^{n}),f(U^{n+1}),f(U^{n+1})](t+k)t.
    \]
    
    The divided differences are
    \begin{align*}
        [f(U^{n})] &= f(U^{n}), \\
        [f(U^{n}),f(U^{n+1})] &= \frac{f(U^{n+1})-f(U^{n})}{k}, \\
        [f(U^{n}),f(U^{n+1}),f(U^{n+2})] &= \frac{[f(U^{n+1}),f(U^{n+2})]-[f(U^{n}),f(U^{n+1})]}{2k} = 
        \frac{f(U^{n+2}) + f(U^{n}) - 2f(U^{n-1})}{2k^2}.
    \end{align*}
    
    Integrating $p(t)$ from 0 to $k$ gives
    
    \begin{align*}
        \int^k_0p(t)dt &= kf(U^{n}) + \frac{f(U^{n+1})-f(U^{n})}{k}(\half t^2 + kt)\big|_0^k +
        \frac{f(U^{n+2}) + f(U^{n}) - 2f(U^{n-1})}{2k^2}(\frac{1}{3}t^3 + \half t^2k)\big|_0^k \\
        &= kf(U^{n}) + (f(U^{n+1})-f(U^{n}))\frac{3k}{2} +(f(U^{n+2}) + f(U^{n}) - 2f(U^{n+1})) \frac{5k}{12} \\
        &=\frac{k}{12}(-f(U^{n}) + 8f(U^{n+1}) + 5f(U^{n+2})).
    \end{align*}
    
    From this it follows that the coefficients $\beta_0,~\beta_1,~\beta_2$ are $\beta_0 = \frac{-1}{12},~
    \beta_1 =\frac{2}{3},~\beta_2=\frac{5}{12}$, as before.
    
    \end{enumerate}
\end{homeworkProblem}

% Problem 6
\begin{homeworkProblem}
    The initial value problem 
    \begin{equation}\label{ode1}
    \begin{split}
    u'(t) &= u(t)^2 - \sin(t) - \cos^2(t),\\
    u(0) &= 1
    \end{split}
    \end{equation}
    has the solution $u(t) = \cos(t)$. 
    
    Write a computer code (preferably in Python or Matlab) to solve
    problem \eqn{ode1} up to time $T=8$ with various different time
    steps $\Delta t = T / N$, with
    \[
    N = 25,~ 50,~ 100,~ 200,~ 400,~ 800,~ 1600,~ 3200.
    \]
    
    Do this using two different methods:
    \begin{enumerate} 
    \item Forward Euler
    \item The Runge-Kutta method (5.32).
    Note that this should be second order accurate for sufficiently small
    $\Delta t$.  If not, then you might have a bug.
    \end{enumerate} 
    
    Produce a log-log plot of the errors versus
    $\Delta t$, with both plots in the same figure.
    
    \vspace{1cm}
    \textbf{Solution:}

    
    \begin{figure}[h!]
        \centering
        \caption{Forward Euler method}
        \begin{subfigure}[b]{0.49\textwidth}
            \includegraphics[width=\textwidth]{FE_appxN50}
            \caption{The Forward Euler solution with $N=50$}
            \label{fig:FEplot1}
        \end{subfigure}
        \begin{subfigure}[b]{0.49\textwidth}
            \includegraphics[width=\textwidth]{FE_error}
            \caption{Log-log plot of the error in Forward Euler}
            \label{fig:FEerror}
        \end{subfigure}
    \end{figure}
    
    As the timestep is decreased in both cases, the error at $t=8$ begins decreasing as $h$ for the Forward
    Euler method and as $h^2$ for the Runge-Kutta method.


    \begin{figure}[h!]
        \centering
        \caption{2-stage Runge-Kutta method}
        \begin{subfigure}[b]{0.49\textwidth}
            \includegraphics[width=\textwidth]{RK_appxN50}
            \caption{The Runge-Kutta solution with $N=50$}
        \end{subfigure}
        \begin{subfigure}[b]{0.49\textwidth}
            \includegraphics[width=\textwidth]{RK_error}
            \caption{Log-log plot of the error in 2-stage Runge-Kutta}
        \end{subfigure}
    \end{figure}
    
\end{homeworkProblem}


\end{document}


